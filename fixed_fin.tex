\documentclass{article}
\usepackage{amsmath}

\sloppy\begin{document}

\section{Introduction to Java Programming}

This document summarizes a tutorial on starting to code with Java.  The tutorial covers installing necessary software and writing a basic "Hello, World!"-style program.

\subsection{Setting up your Java Development Environment}

To begin coding in Java, you need two key components:

\begin{itemize}
    \item \textbf{Java Development Kit (JDK)}: This contains the compiler, which translates your code into machine-readable bytecode. Download the JDK from Oracle (link to be provided in the original video's description). The tutorial uses JDK 23 as an example.  The specific version may vary.
    \item \textbf{Integrated Development Environment (IDE)}:  This provides a workspace for writing and running code. The tutorial recommends IntelliJ IDEA Community Edition (free version).
\end{itemize}

The tutorial details the installation process for both the JDK (including selecting the appropriate installer for your operating system) and IntelliJ IDEA.  It also notes the option to update path variables, but it's not necessary for this specific tutorial.


\subsection{Creating Your First Java Program}

After installing the JDK and IDE, the tutorial guides you through creating a new project in IntelliJ IDEA. Key steps include:

\begin{itemize}
    \item Creating a new project (e.g., "My First Project").
    \item Selecting the appropriate JDK version.
    \item Creating a main Java file (e.g., "Main.java").
    \item Adjusting font size in IntelliJ IDEA settings (using \texttt{Control} + mouse wheel).
\end{itemize}

The core of the program is the \texttt{main} method:

\texttt{public static void main(String[] args) \{\}}

This method is essential for running Java code.  The tutorial explains that while the meaning of the keywords will be covered later, it's a necessary component for program execution.

To print output to the console, the tutorial uses:

\texttt{System.out.println("I like pizza");}

This line prints the text "I like pizza" to the console. Running the program (via the green arrow or \texttt{Shift} + \texttt{F10}) will show this output, confirming successful program execution.  Error handling is briefly touched on; a successful run shows "processed finished with exit code zero".


```latex
\section{First Java Program}

\subsection{Basic Output}

\begin{itemize}
    \item Using \texttt{system.out.print} displays output on the same line.  For example: \texttt{system.out.print("I like pizza");}
    \item Using \texttt{system.out.println} (or \texttt{print Ln} which is an escape sequence) moves the output to the next line.
    \item Comments are added using \texttt{//} for single-line comments and \texttt{/* ... */} for multi-line comments.  These are ignored by the compiler.
\end{itemize}

\subsection{Console Customization}

\begin{itemize}
    \item In IntelliJ, you can customize the console font and colors under \texttt{File} $\rightarrow$ \texttt{Settings} $\rightarrow$ \texttt{Color Scheme} $\rightarrow$ \texttt{Console Font}.
\end{itemize}

\subsection{Shortcuts}

\begin{itemize}
    \item Typing \texttt{sout} and pressing Tab in IntelliJ auto-generates a \texttt{system.out.println} statement.
\end{itemize}

\subsection{Homework}

Post three \texttt{println} statements (e.g., a poem or song lyrics) in the comments section.


\section{Variables in Java}

\subsection{Introduction to Variables}

A variable is a reusable container for a value.  Variables can hold numbers, characters, and words.  There are different data types, including:

\begin{itemize}
    \item \textbf{Integers (\texttt{int}):} Whole numbers.
    \item \textbf{Doubles (\texttt{double}):} Numbers with decimals.
\end{itemize}

\subsection{Variable Types}

There are two categories of variables:

\begin{itemize}
    \item \textbf{Primitive variables:} Simple values stored directly in memory (stack).
    \item \textbf{Reference variables:} Hold memory addresses (stack) that point to locations in the heap.  Think of it like an IOU.
\end{itemize}

\subsection{Creating Variables}

Creating a variable involves two steps:

\begin{enumerate}
    \item \textbf{Declaration:} Specify the data type and variable name (e.g., \texttt{int age;}).
    \item \textbf{Assignment:} Assign a value to the variable (e.g., \texttt{age = 21;}).
\end{enumerate}

\subsection{Integer Example}

The code \texttt{int age = 21;} declares an integer variable named \texttt{age} and assigns it the value 21.  \texttt{system.out.println(age);} would then print 21 to the console.

%\end{itemize}


**Summary:** The text explains the basics of creating a simple Java program, including printing to the console, using comments, and customizing the console appearance.  It then introduces the concept of variables in Java, differentiating between primitive and reference types, and illustrating how to declare and assign values to integer variables.
```


\section{Data Types in Programming}

This text summarizes basic data types in a programming language (likely Java, given the `system.out.println` reference), focusing on integers, doubles, characters, and booleans.  It also briefly introduces reference data types like strings.

\subsection{Integers (\texttt{int})}

Integers are whole numbers.  Examples include:

\begin{itemize}
    \item Years (e.g., 2025)
    \item Quantities (e.g., 1)
\end{itemize}

\subsection{Doubles (\texttt{double})}

Doubles are numbers that can contain decimal portions. Examples include:

\begin{itemize}
    \item Prices (e.g., \$19.99)
    \item GPAs (e.g., 3.5)
    \item Temperatures (e.g., 12.5)
\end{itemize}

Assigning an integer to a double variable will append a \texttt{.0} to the integer value, making it a double.

\subsection{Characters (\texttt{char})}

Characters are single characters enclosed in single quotes.  Examples include:

\begin{itemize}
    \item Letter grades (e.g., \texttt{'A'})
    \item Symbols (e.g., \texttt{'!'})
    \item Currency symbols (e.g., \texttt{'\$'})
\end{itemize}

\subsection{Booleans (\texttt{boolean})}

Booleans represent true or false values. Examples include:

\begin{itemize}
    \item \texttt{isStudent} (true/false)
    \item \texttt{forSale} (true/false)
    \item \texttt{isOnline} (true/false)
\end{itemize}

Booleans are often used in conditional statements (like \texttt{if} statements) to control program flow.  The text shows a basic example using an \texttt{if-else} statement to check the value of a boolean variable.

\subsection{Strings}

Strings are a sequence of characters.  (Further detail is omitted from the provided text).


The text emphasizes that these are basic data types; more advanced types (floats, long doubles) exist but aren't necessary for beginners.


\section{Introduction to Variables in Java}

\subsection{Strings}
Objects are complex; we'll revisit them later.  We'll create strings (series of characters).  Examples include names, emails, and car models. Strings are enclosed in double quotes, while characters are in single quotes.

\begin{itemize}
    \item \texttt{String name = "Bro Code";}
    \item \texttt{String food = "pizza";}
    \item \texttt{String email = "fake123@gmail.com";}
    \item \texttt{String car = "Mustang";}
    \item \texttt{String color = "red";}
\end{itemize}

Strings can contain numbers, but are treated as characters, not numerical values.  String concatenation combines strings using the \texttt{+} operator.  For example: \texttt{System.out.println("Hello " + name);}

The font color of strings often differs from primitive data types (booleans, chars, doubles, ints) in IDEs.


\subsection{Variable Usage and Examples}

\begin{itemize}
    \item Printing variables:  \texttt{System.out.println("Your age is " + age + " years old");}
    \item Combining multiple variables in a single \texttt{println} statement:  \texttt{System.out.println("Your choice is a " + color + " " + year + " " + car);}
    \item Using a boolean variable:  \texttt{if (forSale) \{ System.out.println("There is a " + car + " for sale"); \} else \{ System.out.println(car + " is not for sale"); \}}
\end{itemize}


\subsection{Variable Types}

There are two categories of variables:

\begin{itemize}
    \item \textbf{Primitive data types:} Simple values (integers, floats, characters, booleans) stored directly in memory (stack).
    \item \textbf{Reference variables:} More complex (strings); store a memory address pointing to a location in the heap.
\end{itemize}

\subsection{Homework Assignment}

Create five variables: a string, an integer, a double, a char, and a boolean.


\section{User Input in Java}

To accept user input, we use the \texttt{Scanner} class.  We need to import it:  \texttt{import java.util.Scanner;}.

The \texttt{Scanner} object allows us to accept input, process it, and produce output.  A typical programming flow involves accepting input, processing it, and then producing output.  This is why accepting user input is important.  We want to do something with the input.


This document summarizes a Java tutorial covering variables and user input.  It explains string manipulation, variable types (primitive and reference), and the use of the \texttt{Scanner} class for user input.  A homework assignment is included.


\section{User Input in Java using Scanner}

This document summarizes how to accept various data types (strings, integers, doubles, booleans) as user input in Java using the `Scanner` class.  It emphasizes proper resource management (closing the `Scanner`) and demonstrates different input methods.

\subsection{Scanner Basics and Resource Management}

\begin{itemize}
    \item Create a `Scanner` object:  \texttt{Scanner scanner = new Scanner(System.in);}
    \item Close the `Scanner` at the end of your program: \texttt{scanner.close();} (Essential to prevent resource leaks).
\end{itemize}

\subsection{Accepting Different Data Types}

\subsubsection{Strings}

\begin{itemize}
    \item To read a line of text (including spaces): \texttt{String name = scanner.nextLine(); }
    \item To read a single word (stops at the first space): \texttt{String firstName = scanner.next(); }
\end{itemize}

\texttt{System.out.print("Enter your name: ");}  followed by  \texttt{String name = scanner.nextLine();} will prompt the user for their name and store it in the \texttt{name} variable.  Outputting the name: \texttt{System.out.println("Hello " + name);}

\subsubsection{Integers}

\begin{itemize}
    \item To read an integer: \texttt{int age = scanner.nextInt(); }
    \item Error handling is required if the user inputs a non-integer value.
\end{itemize}

Example: \texttt{System.out.print("Enter your age: ");} followed by \texttt{int age = scanner.nextInt();} and then \texttt{System.out.println("You are " + age + " years old");}


\subsubsection{Doubles}

\begin{itemize}
    \item To read a double (floating-point number): \texttt{double gpa = scanner.nextDouble(); }
\end{itemize}

Example:  \texttt{System.out.print("What is your GPA? ");} followed by \texttt{double gpa = scanner.nextDouble();} and then \texttt{System.out.println("Your GPA is " + gpa);}


\subsubsection{Booleans}

\begin{itemize}
    \item To read a boolean (true/false): \texttt{boolean isStudent = scanner.nextBoolean(); }
\end{itemize}

Example: \texttt{System.out.print("Are you a student (true/false)? ");} followed by  \texttt{boolean isStudent = scanner.nextBoolean();} and then using an \texttt{if} statement to handle the boolean value (e.g., to print different messages based on whether the user is a student).  This example demonstrates the usage of an \texttt{if} statement, which is explained further in the text but not detailed here.


\subsection{Common Issues and Best Practices}

\begin{itemize}
    \item  Mixing \texttt{next()} and \texttt{nextLine()}:  Be mindful of how \texttt{next()} leaves a newline character in the input buffer, potentially causing unexpected behavior when following a \texttt{next()} call with a \texttt{nextLine()} call.
    \item Error Handling: Include error handling (e.g., \texttt{try-catch} blocks) when reading numbers to gracefully handle cases where the user enters invalid input.
\end{itemize}

This summary provides a concise overview of the Java `Scanner` class functionalities for user input and highlights essential aspects for efficient and robust code.  More advanced concepts like \texttt{if} statements and detailed error handling are mentioned but not explicitly coded here.


\section{Java Input and Output}

\subsection{Handling Newline Characters}

This section discusses a common issue in Java when reading user input: the newline character left in the input buffer after reading an integer or double before reading a string.  This leads to unexpected behavior.

\begin{itemize}
    \item The problem arises when using \texttt{next()} for integers/doubles and \texttt{nextLine()} for strings. The \texttt{nextLine()} method consumes the newline character left behind by the previous input.
    \item Solution:  After reading an integer or double, call \texttt{scanner.nextLine()} without assigning it to a variable to clear the newline character from the buffer.
\end{itemize}

\subsection{Calculating the Area of a Rectangle}

This section describes a program that calculates the area of a rectangle based on user input.

\begin{itemize}
    \item Declare variables: \texttt{double width = 0;}, \texttt{double height = 0;}, \texttt{double area = 0;}.
    \item Use a \texttt{Scanner} object to get user input for width and height using \texttt{nextDouble()}.
    \item Calculate the area: \texttt{area = width * height;}.
    \item Output the result: \texttt{System.out.println("The area is " + area + " centimeters\textsuperscript{2}");}.  Note the use of \texttt{\textbackslash{}textsuperscript{2}} for the superscript.
\end{itemize}

\subsection{Mad Libs Game}

This section outlines a Mad Libs game implemented in Java.  Mad Libs is a game where a user provides words (adjectives, nouns, verbs) to fill in blanks in a story.

\begin{itemize}
    \item Import \texttt{java.util.Scanner}.
    \item Declare string variables: \texttt{adjective1}, \texttt{noun1}, \texttt{adjective2}, \texttt{verb1}, \texttt{adjective3}.
    \item Use a \texttt{Scanner} to prompt the user for input for each variable using \texttt{nextLine()}.
    \item Construct the Mad Libs story using string concatenation, incorporating the user-provided words.
\end{itemize}

\section{Summary}

The text provides examples of basic Java input/output operations. It highlights the issue of newline characters in the input buffer and demonstrates how to solve it.  It then gives examples of programs to calculate a rectangle's area and to create a simple Mad Libs game, showcasing the use of \texttt{Scanner} for user input and string manipulation for output.


\section{Mad Libs Game in Java}

This section describes a simple Mad Libs game implemented in Java.  The program prompts the user for various words (adjectives, nouns, verbs) and then generates a short story using these inputs.

\subsection{Program Logic}

The core logic involves:

\begin{itemize}
    \item Prompting the user to enter an adjective, noun, another adjective, a verb (present tense, ending in -ing), and a final adjective.
    \item Using a \texttt{Scanner} to read user input.
    \item Storing the inputs in variables (\texttt{adjective1}, \texttt{noun1}, \texttt{adjective2}, \texttt{verb1}, \texttt{adjective3}).
    \item Concatenating the inputs to create the Mad Libs story.
    \item Closing the \texttt{Scanner} to avoid resource leaks.
    \item Displaying the generated story.
\end{itemize}

\subsection{Example Interaction}

An example interaction might look like this:

\begin{itemize}
    \item \textbf{Prompt:} Enter an adjective
    \item \textbf{User Input:} suspicious
    \item \textbf{Prompt:} Enter a noun
    \item \textbf{User Input:} Mark Zuckerberg
    \item \textbf{Prompt:} Enter an adjective
    \item \textbf{User Input:} smelly
    \item \textbf{Prompt:} Enter a verb ending with -ing
    \item \textbf{User Input:} screaming
    \item \textbf{Prompt:} Enter an adjective
    \item \textbf{User Input:} happy
\end{itemize}

This would result in a story like:  "I went to a suspicious zoo. In an exhibit I saw a Mark Zuckerberg. Mark Zuckerberg was smelly and screaming. I was happy."


\section{Arithmetic Operators in Java}

This section covers basic arithmetic operators in Java, including augmented assignment operators and increment/decrement operators.

\subsection{Basic Arithmetic}

The basic arithmetic operators are demonstrated using variables \texttt{x}, \texttt{y}, and \texttt{z}:

\begin{itemize}
    \item \textbf{Addition:} \texttt{z = x + y} (e.g., 10 + 2 = 12)
    \item \textbf{Subtraction:} \texttt{z = x - y} (e.g., 10 - 2 = 8)
    \item \textbf{Multiplication:} \texttt{z = x \* y} (e.g., 10 \* 2 = 20)
    \item \textbf{Division:} \texttt{z = x / y} (e.g., 10 / 2 = 5. Note integer division truncates decimals.)
    \item \textbf{Modulus:} \texttt{z = x \% y} (e.g., 10 \% 2 = 0; 10 \% 3 = 1. Returns the remainder of the division.)
\end{itemize}

\subsection{Augmented Assignment Operators}

These operators provide a shorthand for combining an arithmetic operation with an assignment:

\begin{itemize}
    \item \texttt{x += y} (equivalent to \texttt{x = x + y})
    \item \texttt{x -= y} (equivalent to \texttt{x = x - y})
    \item \texttt{x \*= y} (equivalent to \texttt{x = x \* y})
    \item \texttt{x /= y} (equivalent to \texttt{x = x / y})
    \item \texttt{x \%= y} (equivalent to \texttt{x = x \% y})
\end{itemize}

\subsection{Increment and Decrement Operators}

These operators increment or decrement a variable by 1:

\begin{itemize}
    \item \texttt{x++} (increments \texttt{x} by 1)
    \item \texttt{x--} (decrements \texttt{x} by 1)
\end{itemize}


\textbf{Summary:} The text describes a simple Mad Libs game implemented in Java and explains basic arithmetic operations, augmented assignment operators, and increment/decrement operators in Java.  It highlights the importance of closing resources (like the \texttt{Scanner}) and points out the effects of integer division.


\section{Order of Operations}

The order of operations follows the acronym PEMDAS (Parentheses, Exponents, Multiplication, Division, Addition, Subtraction).  You might also hear "Please Excuse My Dear Aunt Sally" or, as the author prefers, "Please Excuse My Dope Ass Swag".

Let's consider the equation:

\texttt{double result = 3 + 4 * (7 - 5) / 2.0;}

Following PEMDAS, we solve the equation from left to right:

\begin{itemize}
    \item \textbf{Parentheses}: (7 - 5) = 2
    \item \textbf{Multiplication}: 4 * 2 = 8
    \item \textbf{Division}: 8 / 2.0 = 4
    \item \textbf{Addition}: 3 + 4 = 7
    \item \textbf{Subtraction}: (There is no subtraction in this equation)
\end{itemize}

Therefore, the result is 7.0.


\section{Shopping Cart Program}

This section details the creation of a simple shopping cart program in Java.

\subsection{Setup and User Input}

The program utilizes the \texttt{Scanner} class to accept user input.  The necessary import statement is:

\texttt{import java.util.Scanner;}

A \texttt{Scanner} object is created:

\texttt{Scanner scanner = new Scanner(System.in);}

Remember to close the scanner when finished:

\texttt{scanner.close();}

Variables are declared to store item name (\texttt{String item}), price (\texttt{double price}), quantity (\texttt{int quantity}), currency (\texttt{char currency}), and total cost (\texttt{double total}).


\subsection{Program Logic}

The program prompts the user for the item name, price, and quantity using \texttt{System.out.print} and retrieves the input using \texttt{scanner.nextLine()}, \texttt{scanner.nextDouble()}, and \texttt{scanner.nextInt()}, respectively.  The total cost is calculated as \texttt{price * quantity}.

\subsection{Output}

The program then outputs a summary of the purchase, including the quantity, item name, price, and total cost.  Example output might look like:

"You have bought 3 pizzas. Your total is \$14.97."


The program demonstrates basic arithmetic operations and user input handling in Java.  The use of \textbackslash{}System.out.print and \textbackslash{}System.out.println for output is shown.  Error handling and more sophisticated input validation are not included in this simplified example.


```latex
\section{Summary}

This text describes a Java program that uses `if`, `else if`, and `else` statements to demonstrate conditional logic.  The program first shows a simple example checking if a user's age is greater than or equal to 18, then expands to include multiple conditions (age ranges and a check for negative age) and user input using the `Scanner` class. Finally, it briefly mentions extending the logic to handle string input (user's name).


\section{Shopping Cart Program (Example)}

A brief example of a simple shopping cart program in Java is mentioned as an exercise.  The example involves calculating the total cost of multiple pizzas.


\section{If Statements in Java}

\subsection{Basic If Statement}

\begin{itemize}
    \item An \texttt{if} statement executes a block of code if its condition is true.
    \item Boolean variables are used to control the execution of the \texttt{if} statement.
    \item Example:  Checking if age is greater than or equal to 18.  If true, print "You are an adult."
\end{itemize}

\subsection{Adding \texttt{else} and \texttt{else if} Clauses}

\begin{itemize}
    \item An \texttt{else} clause provides an alternative block of code to execute if the \texttt{if} condition is false.
    \item \texttt{else if} clauses allow for multiple conditional checks.
    \item Example:  Adding checks for age ranges (child, baby, senior citizen) using \texttt{else if}.  The \texttt{else} clause handles cases where none of the previous conditions are met.
\end{itemize}

\subsection{Order of Conditions}

\begin{itemize}
    \item The order of \texttt{if} and \texttt{else if} statements matters.  Conditions are checked sequentially from top to bottom.
    \item If multiple conditions are true, only the code associated with the first true condition is executed.
\end{itemize}

\subsection{User Input with Scanner}

\begin{itemize}
    \item The program is modified to take user input for age using the \texttt{java.util.Scanner} class.
    \item  \texttt{scanner.nextInt()} is used to read integer input.
    \item  It's important to close the scanner after use (\texttt{scanner.close()}).
\end{itemize}

\subsection{String Input}

\begin{itemize}
    \item The program is briefly mentioned to be expandable to handle string input (user's name) using \texttt{scanner.nextLine()}.
\end{itemize}

\subsection{Comparison Operator}

\begin{itemize}
    \item  The double equals sign (\texttt{==}) is used for comparison, while a single equals sign (\texttt{=}) is for assignment.
\end{itemize}
```


\section{If-Else Statements in Java}

\subsection{Group 1: Name Input}

\begin{itemize}
    \item Checks if the name variable is empty using \texttt{name.isEmpty()}.
    \item If empty, prints "You didn't enter your name".  Adds a pouting face emoji.
    \item If not empty, prints "Hello \textit{name}!". Adds a smiley face emoji.
\end{itemize}

\subsection{Group 2: Age Input}

\begin{itemize}
    \item  Checks the user's age.  (Details of age checking logic not explicitly provided in text).
    \item Outputs age-related messages (e.g., "you are an adult", "you haven't been born yet").  Adds relevant emojis.
\end{itemize}

\subsection{Group 3: Student Status}

\begin{itemize}
    \item Uses a Boolean variable \texttt{isStudent} to store user input.
    \item  If \texttt{isStudent} is true, prints "You are a student". Adds a school emoji.
    \item If \texttt{isStudent} is false, prints "You are not a student". Adds an office emoji.
\end{itemize}

\subsection{Summary}
The code uses three groups of if-else statements to handle user input: name, age, and student status.  Each group uses conditional logic and outputs different messages and emojis based on the user's responses.


\section{Generating Random Numbers in Java}

\subsection{Importing and Creating a Random Object}

\begin{itemize}
    \item Imports the \texttt{java.util.Random} class.
    \item Creates a \texttt{Random} object using \texttt{Random random = new Random();}.
\end{itemize}

\subsection{Generating Random Integers}

\begin{itemize}
    \item Uses the \texttt{nextInt(bound)} method to generate random integers.
    \item \texttt{nextInt(bound)} generates a random integer between 0 (inclusive) and \texttt{bound} (exclusive).  For example, \texttt{nextInt(7)} generates numbers from 0 to 6.
    \item To get a random number in a range [a, b], use \texttt{a + random.nextInt(b - a + 1)}.
\end{itemize}

\subsection{Generating Random Doubles}

\begin{itemize}
    \item The text mentions the ability to generate random doubles but does not provide code examples.
\end{itemize}

\subsection{Summary}
The code demonstrates how to use the \texttt{java.util.Random} class to generate random integers within specified bounds.  The example shows how to generate single and multiple random numbers.  The capability of generating random doubles is also noted.


\section{Generating Random Numbers and Using the Random Module in Java}

\subsection{Generating Random Numbers}

\begin{itemize}
    \item Change the data type to \texttt{double}.
    \item Use the \texttt{nextDouble()} method to generate a random number between 0 and 1.
    \item Generate a random boolean value using \texttt{nextBoolean()}.  This can be used to simulate a coin flip.
    \item Example:  Simulate a coin flip using an \texttt{if} statement to display "Heads" or "Tails" based on a random boolean value.
\end{itemize}

\section{Useful Math-Related Constants and Methods in Java}

\subsection{Constants}

\begin{itemize}
    \item \texttt{Math.PI}:  Provides the value of $\pi$.
    \item \texttt{Math.E}: Provides the value of Euler's number ($e$).
\end{itemize}

\subsection{Methods}

\begin{itemize}
    \item \texttt{Math.pow(base, exponent)}: Raises a \texttt{base} to a given \texttt{exponent}.
    \item \texttt{Math.abs(number)}: Returns the absolute value of a number.
    \item \texttt{Math.sqrt(number)}: Returns the square root of a number.
    \item \texttt{Math.round(number)}: Rounds a number to the nearest whole integer.
    \item \texttt{Math.ceil(number)}: Rounds a number up to the nearest integer.
    \item \texttt{Math.floor(number)}: Rounds a number down to the nearest integer.
    \item \texttt{Math.max(number1, number2)}: Returns the maximum of two numbers.
    \item \texttt{Math.min(number1, number2)}: Returns the minimum of two numbers.
\end{itemize}

\section{Math-Related Exercises}

\subsection{Hypotenuse of a Right Triangle}

This exercise involves calculating the hypotenuse of a right triangle using user input for the lengths of the other two sides.  The formula used is $c = \sqrt{a^2 + b^2}$, where $c$ is the hypotenuse, and $a$ and $b$ are the lengths of the other two sides. The solution utilizes the \texttt{Scanner} class for user input, the \texttt{Math.pow()} method for squaring the sides, and the \texttt{Math.sqrt()} method for calculating the square root.  Error handling and resource management (closing the \texttt{Scanner}) are important considerations.  Units of measurement (e.g., centimeters) can be optionally added to the output.

**Summary:** The text describes how to generate random numbers and use the \texttt{random} module in Java, along with a detailed explanation of several useful mathematical constants and methods within the \texttt{Math} class.  A practical exercise demonstrating the calculation of the hypotenuse of a right triangle using user input and several \texttt{Math} class methods is also provided.


\section{Summary}

This text covers two main topics: basic geometric calculations in Java and the use of the `printf` statement for formatted output.  The geometric calculations section demonstrates calculating the circumference, area, and volume of a circle/sphere given its radius, incorporating user input and mathematical formulas. The second section details the `printf` statement, explaining its use as an alternative to `println` and `print`, showcasing its ability to format output by using placeholders and specifiers for various data types.

\section{Geometric Calculations in Java}

\subsection{Calculating Circle/Sphere Properties}

\begin{itemize}
    \item The program prompts the user to enter the radius of a circle or sphere.
    \item It calculates the circumference ($2 \times \pi \times radius$), area ($\pi \times radius^2$), and volume ($\frac{4}{3} \times \pi \times radius^3$) using Java's `Math` class.
    \item It displays the results, including appropriate units (cm, $cm^2$, $cm^3$).
    \item The use of `printf` for formatted output (limiting decimal places) is mentioned as a future topic.
\end{itemize}

\subsection{Code Snippet (Partial)}

The code uses a `Scanner` object to get user input, declares variables (`radius`, `circumference`, `area`, `volume`), and performs calculations using `Math.PI` and `Math.pow()`.  Output is initially done using `println`, with later mention of using `printf` for improved formatting.


\section{The \texttt{printf} Statement in Java}

\subsection{Formatted Output}

\begin{itemize}
    \item \texttt{printf} is an alternative to \texttt{println} and \texttt{print} for formatted output.
    \item It uses placeholders (\texttt{\%}) followed by specifiers (e.g., \texttt{s} for String, \texttt{c} for char, \texttt{d} for int, \texttt{f} for double, \texttt{b} for boolean) to insert variables into the output string.
    \item Example: \texttt{printf("Hello \%s", name);} inserts the value of the \texttt{name} variable into the string.
\end{itemize}

\subsection{Example with Different Data Types}

The example shows how to use \texttt{printf} with variables of different data types (String, char, int, double, boolean) to create a formatted output string.  This demonstrates the flexibility of \texttt{printf} in handling various data types within a single output statement.


\section{Formatting Specifiers in \texttt{printf}}

This text describes how to use format specifiers in Java's \texttt{printf} method for various data types.  It covers inserting variables, handling newlines, precision, and flags for formatting output.

\subsection{Basic Data Types}

\begin{itemize}
    \item \textbf{Strings (\texttt{s}):}  Use \texttt{\%s} as a placeholder, followed by a comma and the string variable. Example: \texttt{printf("hello \%s", variableName)}.
    \item \textbf{Characters (\texttt{c}):} Use \texttt{\%c} as a placeholder, followed by a comma and the character variable. Example: \texttt{printf("Your name starts with a \%c", charVariable)}.
    \item \textbf{Integers (\texttt{d}):} Use \texttt{\%d} as a placeholder, followed by a comma and the integer variable. Example: \texttt{printf("You are \%d years old", ageVariable)}.
    \item \textbf{Doubles/Floats (\texttt{f}):} Use \texttt{\%f} as a placeholder, followed by a comma and the double/float variable. Example: \texttt{printf("You are \%f inches tall", heightVariable)}.
    \item \textbf{Booleans (\texttt{b}):} Use \texttt{\%b} as a placeholder, followed by a comma and the boolean variable. Example: \texttt{printf("Employed: \%b", employedVariable)}.
\end{itemize}

\subsection{Newlines}

Remember to add \texttt{\textbackslash{}n} after the format specifier to insert a new line character in your output.

\subsection{Precision}

To limit the number of digits after the decimal point in floating-point numbers, add a dot followed by the desired number of digits between the percent sign and the \texttt{f}. For example, \texttt{\%.1f} displays one digit after the decimal.

\subsection{Flags}

Several flags can modify the output format:

\begin{itemize}
    \item \texttt{+}: Adds a plus sign before positive numbers.
    \item \texttt{,}: Adds a comma as a thousands separator.
    \item \texttt{(}: Encloses negative numbers in parentheses.
    \item \texttt{ }: Adds a leading space before positive numbers.
\end{itemize}

\subsection{Width}

By specifying a width before the format specifier (e.g., \texttt{\%8d}), you can align integers by padding with spaces.


\subsection{Example Summary}

The text provides a comprehensive guide to using format specifiers in \texttt{printf} statements in Java, covering basic data types, newlines, precision control for floating-point numbers, various flags for enhanced formatting, and width specification for aligning numerical output.  The examples illustrate how to use these features effectively.


\section{Java Output Formatting with \texttt{printf}}

\subsection{Zero Padding and Justification}

This section discusses formatting output in Java using \texttt{printf}.  We can zero-pad numbers to ensure they align vertically.  For example, to pad with four zeros, we use the format specifier `%04d`.  We can also right-justify (\texttt{WR}) using a positive number after the placeholder (e.g., `%4d`) or left-justify using a negative number (e.g., `%-4d`).

\begin{itemize}
    \item Zero padding:  Achieved by adding a zero after the percentage sign in the format specifier (e.g., `%04d` pads to four digits).
    \item Right justification: Use a positive number after the placeholder (e.g., `%4d`).
    \item Left justification: Use a negative number after the placeholder (e.g., `%-4d`).
\end{itemize}


\subsection{The \texttt{printf} Statement}

The \texttt{printf} statement is a method used to format output.  It uses placeholders followed by characters specifying flags, width, precision, and the data type.

\begin{itemize}
    \item Placeholders: Indicate where variables will be inserted.
    \item Flags, width, precision, and specifiers: Control formatting aspects like zero-padding, justification, and the number of decimal places.
\end{itemize}


\section{Compound Interest Calculator Project}

This project implements a compound interest calculator in Java 11.

\subsection{Project Setup}

\begin{itemize}
    \item Imports:  The project starts by importing the \texttt{java.util.Scanner} class.
    \item Scanner object creation: A \texttt{Scanner} object is created to get user input (\texttt{Scanner scanner = new Scanner(System.in)}).  Remember to close the scanner when finished to release resources.
    \item Variable declaration: Variables are declared to store the principal amount (\texttt{double principal}), interest rate (\texttt{double rate}), times compounded (\texttt{int timesCompounded}), number of years (\texttt{int years}), and the final amount (\texttt{double amount}).
\end{itemize}

\subsection{User Input and Calculation}

The program prompts the user for the principal amount, interest rate (as a percentage), times compounded per year, and the number of years.  The interest rate is then adjusted by dividing by 100. The compound interest formula is used to calculate the final amount:

\texttt{amount = principal \* Math.pow(1 + rate / timesCompounded, timesCompounded \* years)}

\subsection{Output}

The calculated amount is displayed using \texttt{System.out.println}.  The example also shows how to use \texttt{printf} to format the output, including specifying the number of decimal places for the final amount and adding currency symbols.  For example, `%.2f` displays a double with two decimal places.

\subsection{Summary}

This document describes using \texttt{printf} for formatted output in Java, including zero-padding and justification.  It then details a compound interest calculator project that demonstrates these concepts, showing user input, calculation, and formatted output using both \texttt{println} and \texttt{printf}.


\section{Compound Interest and Nested If Statements in Java}

A compound interest calculator example is mentioned, yielding \$1,477 after two years.  The main focus, however, is on nested if statements.

\subsection{Nested If Statements}

\begin{itemize}
    \item Java allows nested if statements: if statements within other if statements.
    \item Example: Calculating movie ticket prices with student and senior discounts.
    \item Boolean variables: \texttt{isStudent}, \texttt{isSenior}.
    \item \texttt{double price} = 9.99;
    \item If \texttt{isStudent} is true: 10\% discount (\texttt{price *= 0.9}).
    \item If \texttt{isSenior} is true (nested within the student check): additional 20\% discount (\texttt{price *= 0.7}).
    \item If \texttt{isSenior} is true (nested within the \texttt{else} of the student check): 20\% discount (\texttt{price *= 0.8}).
    \item \texttt{printf} used for formatted output (e.g., \texttt{\$%.2f}).
\end{itemize}

The example demonstrates how nested if statements and else-if structures can handle multiple conditions and discount calculations.  Test runs showcase different scenarios (student only, senior only, both, neither).


\section{String Methods in Java}

\subsection{Useful String Methods}

\begin{itemize}
    \item \texttt{length()} method: Returns the length of a string.  Useful for password validation (checking length limits).
    \item \texttt{charAt()} method: Returns the character at a specified index (starting from 0).
\end{itemize}

The example shows how to use the \texttt{length()} and \texttt{charAt()} methods to access string properties.  The \texttt{length()} method is demonstrated by calculating the length of a string variable (a YouTube channel name in the example), while the \texttt{charAt()} method is used to extract a single character from the string.


\section{Java String Methods}

This document summarizes several useful Java string methods.

\subsection{Character Access and Manipulation}

\begin{itemize}
    \item \textbf{char charAt(int index)}: Returns the character at a specified index.  For example,  \texttt{name.charAt(2)} returns the third character of the string \texttt{name}.
    \item \textbf{int indexOf(int ch)}: Returns the index of the first occurrence of a character.  \texttt{name.indexOf('o')} finds the first 'o' in \texttt{name}.
    \item \textbf{int lastIndexOf(int ch)}: Returns the index of the last occurrence of a character.
    \item \textbf{String toUpperCase()}: Converts a string to uppercase.
    \item \textbf{String toLowerCase()}: Converts a string to lowercase.
    \item \textbf{String trim()}: Removes leading and trailing whitespace.
    \item \textbf{String replace(char oldChar, char newChar)}: Replaces all occurrences of a character with another.
\end{itemize}

\subsection{Boolean Methods}

\begin{itemize}
    \item \textbf{boolean isEmpty()}: Checks if a string is empty.
    \item \textbf{boolean contains(CharSequence s)}: Checks if a string contains a specified sequence of characters.
    \item \textbf{boolean equals(Object anObject)}: Checks if two strings are equal (case-sensitive).
    \item \textbf{boolean equalsIgnoreCase(String anotherString)}: Checks if two strings are equal, ignoring case.
\end{itemize}


\subsection{Substring Method}

\begin{itemize}
    \item \textbf{String substring(int beginIndex, int endIndex)}: Extracts a portion of a string.  \texttt{email.substring(0, 6)} extracts the first six characters of the \texttt{email} string. The \texttt{endIndex} is exclusive.
\end{itemize}

The document covers fundamental string manipulation techniques in Java, including character access, case conversion, whitespace handling, comparison, and substring extraction.  Boolean methods are also discussed for checking string properties such as emptiness and equality.  The examples given demonstrate how to use these methods effectively.


\section{Email Slicer Program}

\subsection{Initial Approach}

The program initially uses the substring method to extract the domain from an email address.  The indices are hardcoded, making the program inflexible for emails of varying lengths.  For example, given the email "user@gmail.com", the code extracts "gmail.com" using a fixed starting index.  This approach is problematic because it won't work correctly with emails of different lengths.

\begin{itemize}
    \item Uses hardcoded indices in the \texttt{substring} method.
    \item Lacks flexibility for emails of varying lengths.
\end{itemize}

\subsection{Improved Approach Using \texttt{indexOf}}

To address the inflexibility, the program is improved by using the \texttt{indexOf} method to dynamically determine the index of the "@" symbol. This index is then used as the starting point for extracting the domain.  The index is incremented by one to exclude the "@" symbol itself.

\begin{itemize}
    \item Uses \texttt{email.indexOf("@")} to find the "@" symbol's index.
    \item Dynamically determines the starting index for the domain.
    \item Improves flexibility by handling emails of different lengths.
\end{itemize}

\subsection{User Input and Validation}

The program is further enhanced by incorporating user input using a \texttt{Scanner} object.  It also includes input validation to check if the email address contains the "@" symbol.

\begin{itemize}
    \item Uses a \texttt{Scanner} to accept user email input.
    \item Includes an \texttt{if} statement to validate the email (checking for "@").
    \item Provides informative error messages for invalid emails.
\end{itemize}

\subsection{Summary}

This program demonstrates the use of Java's string manipulation methods (\texttt{substring} and \texttt{indexOf}) to create an email slicer. It highlights the importance of flexible program design and input validation for robust applications.  The initial hardcoded approach is replaced with a dynamic solution using \texttt{indexOf}, significantly improving the program's adaptability to various email lengths.  The addition of user input and validation further enhances the program's usability and reliability.


\section{Weight Converter Program}

\subsection{Program Overview}

This program converts weights between pounds and kilograms. It utilizes \texttt{if} statements to handle different conversion options chosen by the user. The program structure is well-organized with comments (pseudocode) to explain each step.

\begin{itemize}
    \item Converts weights between pounds and kilograms.
    \item Uses \texttt{if} statements for conditional logic.
    \item Includes comments (pseudocode) for clarity.
    \item Employs a \texttt{Scanner} for user input.
\end{itemize}

\subsection{Variables and Logic}

The program declares variables to store the weight, the conversion choice, and the result.  A welcome message is displayed, followed by a prompt for user input. Based on the user's choice (1 or 2), the appropriate conversion is performed using appropriate formulas.  An \texttt{else} statement handles invalid choices.

\begin{itemize}
    \item Declares variables for weight, choice, and converted weight.
    \item Presents a welcome message and input prompt.
    \item Uses conditional statements (\texttt{if}, \texttt{else}) for different conversion scenarios.
    \item Handles invalid user input.
\end{itemize}

\subsection{Summary}

The weight converter program demonstrates basic input/output, conditional logic, and variable manipulation in Java. The use of comments enhances readability and maintainability.  The program is structured clearly, making it easy to understand and extend.


\section{Weight Conversion Program in Java}

This program converts weight between pounds and kilograms.  The user selects an option (1 for lbs to kgs, 2 for kgs to lbs), enters the weight, and the program outputs the converted weight.

\subsection{Program Logic}

\begin{itemize}
    \item The program prompts the user to choose an option (1 or 2).
    \item User input is received using a \texttt{Scanner}.
    \item An \texttt{if-else if-else} statement handles the chosen option:
    \begin{itemize}
        \item \textbf{Option 1 (lbs to kgs): } The program prompts for weight in pounds, converts it to kilograms using the formula: \texttt{weight\_kgs = weight\_lbs * 0.453592}, and displays the result.  \texttt{printf} is used for formatted output to control the number of decimal places.
        \item \textbf{Option 2 (kgs to lbs): } The program prompts for weight in kilograms, converts it to pounds using the formula: \texttt{weight\_lbs = weight\_kgs * 2.20462}, and displays the result.
        \item \textbf{Invalid Option:} If the user enters a number other than 1 or 2, an error message is displayed.
    \end{itemize}
    \item The \texttt{Scanner} is closed at the end of the program to release resources.
\end{itemize}


\section{Ternary Operator in Java}

The ternary operator provides a concise alternative to \texttt{if-else} statements.  Its syntax is:

\texttt{condition ? value\_if\_true : value\_if\_false}

\subsection{Examples}

\begin{itemize}
    \item \textbf{Pass/Fail Example:}  A score is checked against a passing threshold (60).  The ternary operator assigns "pass" or "fail" to a string variable based on the condition.
    \item \textbf{Even/Odd Example:} A number is checked for evenness using the modulus operator (\texttt{\%}). The ternary operator assigns "even" or "odd" based on whether the remainder after division by 2 is 0 or not.
\end{itemize}

The ternary operator simplifies code by expressing conditional logic in a single line, making it more readable and efficient in many cases.  It's a useful tool for writing shorter, more expressive Java code.


\section{Ternary Operator Examples}

\subsection{Even/Odd Number Check}

Is a number divisible by two? If the remainder is zero, it's even; otherwise, it's odd.  This can be implemented using the ternary operator:

\begin{itemize}
    \item Check if the number modulo 2 equals 0.
    \item If true, return "even".
    \item Otherwise, return "odd".
\end{itemize}

\subsection{Time of Day (AM/PM)}

Given an hour (military time), determine if it's AM or PM:

\begin{itemize}
    \item Check if the hour is greater than or equal to 12.
    \item If true, return "PM".
    \item Otherwise, return "AM".
\end{itemize}

\subsection{Tax Rate Calculation}

Based on income, determine the tax rate:

\begin{itemize}
    \item Check if the income is greater than or equal to \$40,000.
    \item If true, the tax rate is 0.25.
    \item Otherwise, the tax rate is 0.15.
\end{itemize}

\section{Temperature Converter}

This program converts between Celsius and Fahrenheit using the ternary operator and user input.  It utilizes a \texttt{Scanner} to obtain user input and handles uppercase/lowercase input for the unit (C or F).

\begin{itemize}
    \item \textbf{Input:} Obtain temperature (double) and unit (string) from the user.
    \item \textbf{Unit Conversion (Ternary Operator):}\newline
    \texttt{unit.equals("C") ? (temp - 32) * 5 / 9 : (temp * 9 / 5) + 32} \newline This checks if the unit is "C". If true, it converts Fahrenheit to Celsius; otherwise, it converts Celsius to Fahrenheit.
    \item \textbf{Output:} Display the converted temperature.
    \item \textbf{Error Handling:}  The provided text doesn't explicitly include error handling for invalid inputs (e.g., non-numeric temperature).  Robust code would need to incorporate such checks.
\end{itemize}

The program uses \texttt{java.util.Scanner} for user input.  Remember to close the \texttt{Scanner} after use to release resources.  The \texttt{toUpperCase()} method is used to handle both uppercase and lowercase input for the unit.


\textbf{Summary:} The text describes the use of the ternary operator in Java through several examples: determining even/odd numbers, identifying AM/PM time, calculating tax rates based on income brackets, and building a temperature converter.  The ternary operator provides a concise way to express conditional logic. The temperature converter example demonstrates input handling and string manipulation techniques alongside the ternary operator.


\section{Temperature Conversion using Ternary Operator}

This section describes a Java program that converts temperatures between Fahrenheit and Celsius using the ternary operator.

\subsection{Implementation}

The core of the program is the temperature conversion formula implemented using the ternary operator:  `ourTemp * 5 \textbackslash{} 9 + 32`.  The result is assigned to a variable, `newTemp`, and then displayed. The program handles formatting to limit decimal places using `printf`.  Degree symbols are added using the `alt + 0176` key combination (Windows).  The output includes the unit (Celsius or Fahrenheit).

\begin{itemize}
    \item Uses a ternary operator for concise temperature conversion.
    \item Assigns the result to a variable for further processing.
    \item Uses `printf` for formatted output, limiting decimal places.
    \item Includes degree symbol and unit in the output.
\end{itemize}


\section{Enhanced Switches in Java}

This section explains the use of enhanced switches in Java as a more efficient alternative to multiple `if-else if` statements.

\subsection{Problem with Multiple `if-else if` Statements}

Using many `if-else if` statements for checking multiple conditions, such as days of the week, leads to redundant code and reduced efficiency.  The example demonstrates this with a program checking the day of the week.

\subsection{Solution: Enhanced Switches}

Enhanced switches provide a cleaner and more efficient solution.  They use the arrow operator (`->`) to specify the action for each case.  A `default` case handles situations where no matching case is found.


\begin{itemize}
    \item Replaces multiple `if-else if` statements with a more concise structure.
    \item Uses the arrow operator (`->`) to map cases to actions.
    \item Includes a `default` case to handle unmatched inputs.
    \item Improves code readability and efficiency compared to nested `if-else if` blocks.
\end{itemize}


\subsection{Implementation of Enhanced Switch}

The example shows how to use an enhanced switch to process the day of the week.  User input is obtained using a `Scanner` object.  The switch statement checks the input against predefined cases (days of the week), performing different actions based on the input. A default case handles invalid input.


\textbf{Summary:} The text describes two Java programming concepts: using the ternary operator for concise temperature conversion and utilizing enhanced switches to improve the efficiency and readability of conditional logic compared to multiple `if-else if` statements. Both examples include user input and formatted output.


\section{Enhanced Switches in Java and a Simple Calculator Program}

This document summarizes a tutorial on enhanced switches in Java and demonstrates their use in a simple calculator program.

\subsection{Enhanced Switches: A Superior Alternative to `if-else if` Chains}

The tutorial highlights the advantages of using enhanced switches over lengthy `if-else if` statement chains, particularly when dealing with multiple cases that perform similar actions.  The example involves a day-of-the-week program initially implemented with numerous `if-else if` statements.  This is refactored using an enhanced switch to consolidate cases for weekdays and weekends.

\begin{itemize}
    \item Enhanced switches provide a more concise and readable way to handle multiple conditions.
    \item They are a Java 14 feature.
    \item They replace cumbersome `if-else if` structures, improving code clarity and maintainability.
\end{itemize}

The example demonstrates how to consolidate multiple cases in a switch statement using comma separation:

\texttt{case Monday, Tuesday, Wednesday, Thursday, Friday -> \dots}


\subsection{Simple Calculator Program using Enhanced Switches}

The tutorial then guides the creation of a basic calculator program as a practical application of enhanced switches.

\begin{itemize}
    \item The program uses a \texttt{Scanner} to obtain user input for two numbers and an operator (+, -, \*, /, \textasciicircum{}).
    \item It employs an enhanced switch to perform the calculation based on the operator.
    \item The program handles addition, subtraction, multiplication, division, and exponentiation.
    \item  The use of \texttt{scanner.close()} is emphasized for resource management.
    \item Necessary imports (\texttt{import java.util.Scanner}) are highlighted.
    \item Variable declaration (double \texttt{num1}, \texttt{num2}, \texttt{result}; char \texttt{operator}) is shown.
    \item The tutorial mentions addressing potential division-by-zero errors in a future enhancement.
    \item  The \texttt{Math.pow()} method is used for exponentiation.
    \item The issue of uninitialized variables and how to address it is discussed.
\end{itemize}

The code snippet illustrates how to use the enhanced switch to perform the calculation:

\texttt{switch (operator) \{
    case '+' -> result = num1 + num2;
    case '-' -> result = num1 - num2;
    case '*' -> result = num1 * num2;
    case '/' -> result = num1 / num2;
    case '\textasciicircum{}' -> result = Math.pow(num1, num2);
\}}


The tutorial concludes by emphasizing the benefits of using enhanced switches for improved code readability and maintainability in Java.


\section{Calculator Program in Java}

This section summarizes a Java calculator program demonstrating enhanced switch statements and error handling.

\subsection{Basic Functionality}

The program performs basic arithmetic operations (+, -, \*, /, \^{}) based on user input.  Initial implementation had issues:

\begin{itemize}
    \item \textbf{Division by zero:} Resulted in an ``infinity'' output.
    \item \textbf{Invalid operator:} Accepted any input, displaying incorrect results.
\end{itemize}

\subsection{Improved Error Handling}

The program was improved to address these issues:

\begin{itemize}
    \item \textbf{Division by zero check:} An \texttt{if} statement was added within the division case to prevent division by zero, printing "cannot divide by zero" instead.
    \item \textbf{Valid operator check:} A Boolean variable, \texttt{validOperation}, was introduced.  It's set to \texttt{false} if division by zero or an invalid operator is detected.
    \item \textbf{Default case:} A default case was added to the \texttt{switch} statement to handle invalid operators, printing "invalid operator" and setting \texttt{validOperation} to \texttt{false}.
    \item \textbf{Conditional Result Output:} The result is only printed if \texttt{validOperation} remains \texttt{true}.
\end{itemize}

The enhanced program correctly handles invalid inputs and prevents erroneous results.  Cases within the \texttt{switch} statement can span multiple lines, enclosed in curly braces \{\}.


\section{Logical Operators in Java}

This section explains logical operators in Java, focusing on the \texttt{and} operator.

\subsection{The \texttt{and} Operator (\texttt{\&\&})}

The \texttt{and} operator (\texttt{\&\&}) allows checking multiple conditions.  All conditions must be \texttt{true} for the entire expression to evaluate to \texttt{true}.

An example using temperature and weather conditions demonstrates its functionality:

\begin{itemize}
    \item A variable \texttt{temp} represents temperature.
    \item A Boolean variable \texttt{isSunny} indicates whether it's sunny.
    \item The code checks if \texttt{temp} is between 0 and 30 degrees Celsius \texttt{and} if \texttt{isSunny} is \texttt{true}.
    \item Only if both conditions are \texttt{true} is a message ("The weather is good \smiley{}") printed.
\end{itemize}

The example shows that with the \texttt{and} operator, a single \texttt{false} condition makes the entire expression \texttt{false}.  Multiple conditions can be chained together.


\section{Logical Operators in Java}

This document summarizes the use of logical operators (AND, OR, NOT) in Java, illustrated with examples involving weather conditions and username validation.

\subsection{AND and NOT Operators}

Multiple conditions can be checked using logical operators.  The AND operator (`&&`) requires all conditions to be true for the entire expression to be true.

\begin{itemize}
    \item Example:  A code block executes only if the weather is good \textit{and} it is sunny.
\end{itemize}

The NOT operator (`!`) reverses the truth value of a condition.

\begin{itemize}
    \item Example: An \texttt{else if} statement executes if the weather is good but it is \textit{not} sunny (implying it's cloudy).
\end{itemize}

\subsection{OR Operator}

The OR operator (`||`) requires at least one condition to be true for the entire expression to be true.

\begin{itemize}
    \item Example: A code block executes if the temperature is really hot \textit{or} really cold.
\end{itemize}


\subsection{Username Validation Example}

This section demonstrates using logical operators to validate a username based on these rules:

\begin{itemize}
    \item Username length must be between 4 and 12 characters.
    \item Username must not contain spaces or underscores.
\end{itemize}

The example uses Java's \texttt{Scanner} class to obtain user input and the \texttt{String} class's \texttt{length()} and \texttt{contains()} methods for validation.  The code employs \texttt{if} and \texttt{else if} statements with logical operators to check the conditions.  If the username doesn't meet the criteria, appropriate error messages are displayed; otherwise, a welcome message is printed.


\subsection{Summary}

This document explains Java's logical operators: AND (\texttt{\&\&}), OR (\texttt{||}), and NOT (\texttt{!}).  The AND operator requires all conditions to be true, the OR operator requires at least one condition to be true, and the NOT operator inverts the truth value of a condition. These operators are crucial for controlling program flow based on multiple conditions, as demonstrated in the provided examples of weather condition checks and username validation.  The examples highlight the use of \texttt{if}, \texttt{else if}, and \texttt{else} statements in conjunction with these logical operators to handle different scenarios.


\section{While Loops in Java}

This text explains the use of \texttt{while} loops in Java, contrasting them with \texttt{if} statements and highlighting the importance of avoiding infinite loops.  The examples use user input and demonstrate how to prevent unexpected program behavior.

\subsection{From \texttt{if} to \texttt{while}}

The text begins by demonstrating a simple name-input program using an \texttt{if} statement.  The problem with this approach is that if the user doesn't enter a name, the program continues without displaying a name.  This is addressed by replacing the \texttt{if} statement with a \texttt{while} loop:

\begin{itemize}
    \item  The \texttt{while} loop continues to prompt the user for input until a name is entered.
    \item This ensures that the program doesn't proceed until the required input is provided.
\end{itemize}

\subsection{Infinite Loops: A Cautionary Tale}

The crucial point emphasized is the potential for infinite loops.  An example is given: \texttt{while (1 == 1)} which will run indefinitely.  This is because the condition (1 == 1) never changes within the loop.  The text warns against such scenarios, stating that the program might freeze.

\subsection{Controlled Loop Termination}

The text then presents a game example where the user can quit by pressing 'Q'.  The loop continues (\texttt{while (!response.equals("Q"))}) until the user enters 'Q'.  Key points here include:

\begin{itemize}
    \item  Initialization of the \texttt{response} variable.
    \item  Conversion of lowercase 'q' to uppercase 'Q' for robust handling of user input.
    \item  Updating the \texttt{response} variable inside the loop to allow termination.
\end{itemize}

\subsection{Preventing Negative Input}

Finally, the text shows how a \texttt{while} loop can be used to validate user input. In this case, it prevents the user from entering a negative age.  The loop (\texttt{while (age < 0)}) continues to reprompt the user until a non-negative age is entered.


\textbf{Summary:} The text provides a practical tutorial on using \texttt{while} loops in Java, emphasizing the importance of proper condition handling to avoid infinite loops and ensure robust program behavior by validating user input.  It uses clear examples to illustrate how to effectively employ \texttt{while} loops for different purposes, from enforcing user input to creating game loops with controlled termination.


\section{While Loops in Java}

\subsection{Standard While Loop}

The standard while loop checks its condition before executing the code within the loop.  If the condition is false initially, the loop is skipped entirely.  If the condition remains true, the loop continues to execute.  This is particularly useful when handling user input, as it allows for repeated prompts until valid input is received.  For example, in a program requesting a user's age, a while loop can ensure the user inputs a non-negative number.  If a negative number is entered, the loop continues to prompt the user for valid input.


\subsection{Do-While Loop}

A do-while loop is a variation where the code within the loop is executed at least once before the condition is checked.  If the condition is true after the first execution, the loop continues.  This guarantees the code within the loop runs at minimum one time.  Converting a standard while loop to a do-while loop alters the program's flow: the code executes first, and the condition is evaluated only afterward.  A practical example includes prompting a user for a number within a specific range (e.g., 1-10).  A do-while loop ensures the user is prompted at least once, even if their initial input is outside the range.

\subsection{Key Points}

\begin{itemize}
    \item Standard while loops check the condition \textit{before} executing the loop body.
    \item Do-while loops execute the loop body \textit{at least once} before checking the condition.
    \item While loops are essential for handling user input, ensuring valid data is received.
    \item Both standard and do-while loops offer different control flows depending on the specific needs of the program.
\end{itemize}


\section{Number Guessing Game in Java}

This section describes the creation of a number guessing game in Java.  The game utilizes a \texttt{Random} object to generate a random number between 1 and 100 and a \texttt{Scanner} object to accept user input.  The game uses a do-while loop to repeatedly prompt the user for a guess until the correct number is entered.

The program requires the following:

\begin{itemize}
    \item \texttt{java.util.Random} for generating random numbers.
    \item \texttt{java.util.Scanner} for user input.
    \item Variables to store the user's guess, the number of attempts, and the randomly generated number.
\end{itemize}

The core logic involves a do-while loop that continues as long as the user's guess does not match the randomly generated number.  Within the loop, the user is prompted to enter their guess.  The program tracks the number of attempts and provides feedback after each guess.  The use of a do-while loop ensures the game plays at least once.



\textbf{Summary:} The text describes while loops in Java, highlighting the difference between standard while loops and do-while loops.  It then details the creation of a number guessing game using Java, demonstrating the practical application of a do-while loop to create an interactive game.


\section{Number Guessing Game in Java}

This document summarizes a Java program that implements a number guessing game, and then extends it to incorporate more flexible parameters and user feedback.  Later, it introduces Java's `for` loop construct.

\subsection{Basic Number Guessing Game}

The program prompts the user to guess a number between 1 and 10.  It keeps track of the number of attempts.  The initial version lacks feedback on whether the guess is too high or too low.

\begin{itemize}
    \item Uses \texttt{Scanner} to get user input.
    \item Generates a random number between 1 and 10.
    \item Implements a \texttt{while} loop to continue guessing until the correct number is entered.
    \item Counts the number of attempts.
\end{itemize}

A later improvement adds feedback to the user:
\begin{itemize}
    \item Includes \texttt{if-else if-else} statements to check if the guess is too high, too low, or correct.
    \item Provides appropriate messages to the user ("Too low, try again," "Too high, try again," "Correct!").
\end{itemize}

The \texttt{Scanner} is explicitly closed using \texttt{scanner.close()}.

\subsection{Enhanced Number Guessing Game}

The game is enhanced to allow the user to specify the minimum and maximum values for the guessing range:

\begin{itemize}
    \item Introduces \texttt{int Min} and \texttt{int Max} variables to define the range.
    \item Uses \texttt{printf} for formatted output, incorporating \texttt{Min} and \texttt{Max} into the prompt.  The maximum value is handled correctly to be inclusive.
\end{itemize}


\subsection{Introduction to \texttt{for} Loops in Java}

The document concludes with an explanation of \texttt{for} loops in Java.

\begin{itemize}
    \item A \texttt{for} loop executes a block of code a specific number of times.
    \item It differs from a \texttt{while} loop, which can run indefinitely.
    \item The \texttt{for} loop structure has three parts:
    \begin{itemize}
        \item \textbf{Initialization:} Typically, a counter variable (e.g., \texttt{int i = 0}) is declared and initialized.
        \item \textbf{Condition:} A Boolean expression that determines whether the loop continues (e.g., \texttt{i < 10}).
        \item \textbf{Update:} An expression that modifies the counter variable after each iteration (e.g., \texttt{i++}).
    \end{itemize}
    \item An example is provided to print "pizza" 10 times using a \texttt{for} loop.
    \item The loop counter can be printed within the loop body.
\end{itemize}


\section{Introduction to Loops in Java}

The text discusses for loops in Java, illustrating their usage with various examples.  Key aspects covered include loop initialization, conditions, increment/decrement, and user input integration.  Additionally, the `break` and `continue` keywords are explained within the context of loop control.

\subsection{Basic For Loop Structure and Variations}

\begin{itemize}
    \item The fundamental structure of a `for` loop is demonstrated using examples with different starting values (0, 1, 10), increment/decrement steps (1, 2, 3), and loop conditions (`i < 10`, `i <= 10`, `i > 0`).
    \item Looping a specific number of times is shown, highlighting the importance of loop initialization to control starting values (0 vs. 1).
    \item  The concept of incrementing/decrementing by values other than 1 is explained and demonstrated.  Examples using `i += 2`, `i += 3`, `i -= 2`, and `i -= 3` are provided.
\end{itemize}

\subsection{For Loops with User Input}

\begin{itemize}
    \item  Integration of user input using the `Scanner` class is explained, emphasizing the need for proper resource management (closing the `Scanner`).
    \item A `for` loop is created to iterate a number of times determined by user input.  The example shows how to use a `Scanner` to get the number of iterations from the user.
\end{itemize}

\subsection{Countdown Timer Example}

\begin{itemize}
    \item A countdown timer program is implemented using a `for` loop, incorporating a `Thread.sleep()` method to simulate a one-second delay between each countdown number.
    \item The use of `Thread.sleep()` necessitates handling the `InterruptedException`, which is explained.
    \item Modification of the countdown program to accept user input for the countdown duration is shown.
\end{itemize}

\subsection{Break and Continue Keywords}

\begin{itemize}
    \item The `break` keyword is explained as a mechanism to exit a loop prematurely, demonstrated by breaking out of a loop when a specific condition is met.
    \item The `continue` keyword is explained as a mechanism to skip the current iteration and proceed to the next, contrasting its behavior with `break`.
\end{itemize}


\section{Summary}

This document provides a comprehensive introduction to `for` loops in Java, covering basic syntax, variations in loop control, user input integration, and the usage of `break` and `continue` keywords for advanced loop control.  The examples clearly illustrate different scenarios and techniques for utilizing `for` loops effectively in Java programming.


\section{Continue vs. Break Keywords in Java Loops}

This section explains the difference between the \texttt{continue} and \texttt{break} keywords in Java loops.  Using \texttt{continue} skips the current iteration and proceeds to the next, while \texttt{break} exits the loop entirely.  An example shows how \texttt{continue} omits a specific cycle (e.g., skipping the number 5 in a loop printing 1-9).


\section{Nested Loops in Java}

\subsection{Introduction to Nested Loops}

A nested loop is a loop inside another loop.  They are frequently used with matrices or in data structures and algorithms.  Any combination of \texttt{for} and \texttt{while} loops can be nested.

\subsection{Example: Printing Numbers 1-9 Three Times}

A simple example demonstrates printing numbers 1 through 9 three times using nested loops.  Initially, the code is repeated three times, violating the DRY (Don't Repeat Yourself) principle.  A nested loop efficiently replaces this repetition.  The example highlights the importance of using different index variables (e.g., \texttt{i} and \texttt{j}) for nested loops to avoid conflicts.

\begin{itemize}
    \item Nested loops are efficient for handling matrix-like data structures.
    \item Different index variables (e.g., \texttt{i}, \texttt{j}) should be used for nested loops to avoid naming conflicts.
\end{itemize}


\subsection{Mini-Project: Creating a Matrix of User-Specified Symbol}

This section describes a mini-project that creates a matrix of a symbol specified by the user.  The user inputs the number of rows, columns, and the symbol.  Nested loops are used to generate the matrix.  The code includes user input using \texttt{Scanner}, appropriate variable declarations (\texttt{int rows}, \texttt{int columns}, \texttt{char symbol}), and input prompts.  The \texttt{Scanner} is properly closed using \texttt{scanner.close()}. The example demonstrates handling user input and building a matrix using nested loops.

\begin{itemize}
    \item User inputs the number of rows, columns, and the symbol.
    \item Nested loops generate the matrix.
    \item \texttt{Scanner} is used for user input and is properly closed.
\end{itemize}


\textbf{Summary:} This document explains the use of \texttt{continue} and \texttt{break} keywords in Java loops, demonstrates nested loops with examples of printing numbers and creating a user-defined symbol matrix, and emphasizes best practices such as avoiding code repetition and using appropriate variable names.  The examples illustrate how nested loops are powerful tools for handling multi-dimensional data.


```latex
\section{Nested Loops}

The text begins with a discussion of nested loops in Java.  The example shows how nested loops can be used to create a grid (five rows and three columns).

\subsection{Key Points about Nested Loops}

\begin{itemize}
    \item Nested loops consist of one loop inside another.
    \item The outer loop controls rows, while the inner loop controls columns (in the given example).
    \item They are frequently used with matrices and in data structures and algorithms.
\end{itemize}

\section{Methods in Java}

The main focus shifts to explaining methods in Java.  A method is described as a block of reusable code.

\subsection{Why Use Methods?}

\begin{itemize}
    \item \textbf{Reusability:} Write code once and use it multiple times.
    \item \textbf{Efficiency:} Avoids code repetition (follows the DRY principle - Don't Repeat Yourself).
\end{itemize}

The example uses a "Happy Birthday" song to illustrate method creation and usage.  A method called \texttt{happyBirthday} is created to encapsulate the song.

\subsection{Method Definition and Calling}

The text explains how to define a method in Java, including:

\begin{itemize}
    \item Return type (\texttt{void} in this case).
    \item Method name (e.g., \texttt{happyBirthday}).
    \item Parameters (explained later).
    \item The \texttt{static} keyword (explained as necessary for this specific example).
\end{itemize}

Calling a method involves using its name followed by parentheses \texttt{()}.

\subsection{Method Scope and Arguments}

A crucial point is made about method scope: variables declared within one method are not accessible from other methods.  To solve this, the text introduces \textbf{arguments} and \textbf{parameters}.

\begin{itemize}
    \item \textbf{Arguments:} Values or variables passed to a method when calling it.
    \item \textbf{Parameters:} Variables declared within the method's definition to receive the arguments.
\end{itemize}

The example modifies the \texttt{happyBirthday} method to accept a name as an argument, demonstrating how to pass and use arguments within a method.
```


\section{Method Parameter Handling and Return Values}

This document summarizes key concepts of method parameters, return values, and data types in a programming context (likely Java or C++).

\subsection{Passing Arguments to Methods}

\begin{itemize}
    \item Arguments passed to a method must match the parameters in number and data type.  Mismatched parameters result in errors (e.g., "actual and formal argument lists differ in length").
    \item Arguments are comma-separated.
    \item Parameter names within the method can differ from argument names in the method call.  Only data type and order matter.
\end{itemize}

\subsection{Returning Values from Methods}

\begin{itemize}
    \item Methods can return values using the \texttt{return} keyword.
    \item The data type of the returned value must be specified (e.g., \texttt{double}, \texttt{int}, \texttt{String}, \texttt{boolean}).  If no value is returned, use \texttt{void}.
    \item A returned value can be assigned to a variable or printed directly.
    \item  When a method returns a value, the method call is conceptually replaced with the returned value.
\end{itemize}


\subsection{Examples}

\begin{itemize}
    \item A method to square a number (\texttt{double square(double number)}) returns the square of the input.
    \item A method to cube a number (\texttt{double cube(double number)}) returns the cube of the input.
    \item A method to create a full name (\texttt{String getFullName(String first, String last)}) concatenates first and last names and returns the result.
    \item An age verification method (\texttt{boolean ageCheck(int age)}) returns \texttt{true} if the age is 18 or older, \texttt{false} otherwise.  This uses an \texttt{if} statement (not shown in detail in the provided text).
\end{itemize}

\subsection{Static Keyword}

The text mentions the need to use the \texttt{static} keyword when calling a non-static method from a static context.  Further details on the \texttt{static} modifier are promised in future content.  This is likely relevant to the class structure of the code.

The example code snippets utilize \texttt{printf} style formatting for output, indicating a C-style language or a language with similar formatting capabilities.  The examples focus on demonstrating parameter passing, return values, and basic data type handling within methods.


\section{Methods and Age Verification}

This section describes a simple method for verifying a user's age and demonstrates its usage.

\subsection{Age Verification Method}

The initial method checks if a user's age is 18 or older.  It can be implemented using an \texttt{if} statement or directly within the return statement:

\begin{itemize}
    \item \textbf{If-statement approach:}
    \begin{verbatim}
if (age >= 18) {
    return true;
} else {
    return false;
}
    \end{verbatim}
    \item \textbf{Concise return approach:}
    \begin{verbatim}
return (age >= 18);
\end{verbatim}
\end{itemize}

This method is then used in an example to determine if a user can sign up:

\begin{verbatim}
int age = 21; 
if (checkAge(age)) {
    System.out.println("You may sign up");
} else {
    System.out.println("You must be 18+ to sign up");
}
\end{verbatim}


\section{Overloaded Methods in Java}

This section explains overloaded methods in Java.

\subsection{Definition and Example}

Overloaded methods in Java share the same name but have different parameter lists. The method's name and parameters together form a unique signature.  Each method signature must be unique; methods can share the same name but must have distinct parameter lists.

For example, an \texttt{add} method can be overloaded to handle different numbers of arguments:

\begin{verbatim}
double add(double a, double b) { return a + b; }
double add(double a, double b, double c) { return a + b + c; }
double add(double a, double b, double c, double d) { return a + b + c + d; }
\end{verbatim}

\subsection{Pizza Baking Example}

Another example demonstrates overloading using a \texttt{bakePizza} method.  Different versions of the method accept different ingredients:

\begin{verbatim}
static String bakePizza(String bread) { return bread + " pizza"; }
static String bakePizza(String bread, String cheese) { return cheese + " " + bread + " pizza"; }
static String bakePizza(String bread, String cheese, String topping) { return topping + " " + cheese + " " + bread + " pizza"; }
\end{verbatim}

This shows how overloaded methods can handle variations in input, resulting in different outputs (e.g., "flatbread pizza," "mozzarella flatbread pizza," "pepperoni mozzarella flatbread pizza").  The correct method is chosen based on the number and types of arguments provided during the method call.


\subsection{Summary}

Overloaded methods provide a way to create multiple methods with the same name but different functionalities based on input parameters, improving code readability and reusability.  The compiler selects the correct method based on the method signature (name and parameter list).


\section{Variable Scope in Java}

\subsection{Local Variables}
\begin{itemize}
    \item A variable declared inside a method has \textbf{local scope}.
    \item It is only accessible within the method where it is declared.
    \item Variables with the same name can exist in different methods without conflict.
    \item  Methods are unaware of variables declared in other methods (like "neighbors who can't see into each other's houses").
\end{itemize}

\subsection{Class Variables}
\begin{itemize}
    \item A variable declared inside a class, but outside any method, has \textbf{class scope}.
    \item It is accessible from any method within the class.
    \item  In our analogy, it's like a variable "on the street," visible to all "houses" (methods).
\end{itemize}

\subsection{Local vs. Class Variables}
\begin{itemize}
    \item If local and class variables share a name, local variables take precedence.
    \item Local variables are generally preferred unless you need a constant or are working with object-oriented programming concepts.
\end{itemize}


\section{Simple Banking Program in Java}

This section describes the steps involved in creating a simple banking program in Java as a beginner project.  The program will allow users to deposit, withdraw, and check their balance.  The project is broken down into manageable steps:

\begin{enumerate}
    \item \textbf{Declare Variables:} Define variables such as `balance` (a `double` to accommodate decimal values) and a `Scanner` for user input.
    \item \textbf{Display Menu:} Present a menu to the user with options for deposit, withdrawal, balance check, and exit.
    \item \textbf{Get and Process User Choice:} Obtain the user's choice from the menu.
    \item \textbf{Create Methods:} Implement separate methods for:
    \begin{itemize}
        \item `showBalance()`: Displays the current account balance.
        \item `deposit()`: Processes a deposit transaction.
        \item `withdraw()`: Processes a withdrawal transaction.
    \end{itemize}
    \item \textbf{Exit Message:} Display a closing message when the user chooses to exit.
\end{enumerate}


\textbf{Summary:} The text describes Java variable scope (local and class) and outlines a step-by-step approach to building a simple banking application in Java for beginners.  The banking program utilizes methods to modularize functionality, improving code organization and readability.


\section{Java Banking Program}

This program simulates a simple banking application using Java.  Key features include showing balance, depositing, withdrawing, and exiting.  The program uses a \texttt{Scanner} to get user input and a \texttt{while} loop to continue operation until the user chooses to exit.

\subsection{Program Structure}

\begin{itemize}
    \item Imports \texttt{java.util.Scanner}.
    \item Initializes variables: \texttt{Scanner}, \texttt{balance} (initially 0), and a Boolean \texttt{isRunning} (set to \texttt{true}).
    \item Uses a \texttt{while} loop (\texttt{while (isRunning)}) to repeatedly present a menu to the user.
    \item Presents a menu with options to show balance, deposit, withdraw, and exit.
    \item Uses a \texttt{switch} statement to process the user's choice.
    \item Includes error handling for invalid choices.
    \item Contains separate methods for showing balance, depositing, and withdrawing (only \texttt{showBalance} is fully implemented in the provided text).
    \item Closes the \texttt{Scanner} (although the code for this is not explicitly shown in the text).
\end{itemize}

\subsection{Menu and User Interaction}

The program displays a menu with the following options:

\begin{enumerate}
    \item Show Balance
    \item Deposit
    \item Withdraw
    \item Exit
\end{enumerate}

The user enters their choice, and the program processes it using a \texttt{switch} statement.

\subsection{Show Balance Method}

A \texttt{static} method \texttt{showBalance} is defined to display the current balance.  It takes a \texttt{double} representing the balance as a parameter and uses \texttt{printf} to format the output, displaying two decimal places using \texttt{\%0.2f}.  The method also includes separators for improved readability.

\subsection{Summary}

This Java program demonstrates basic console input/output, control flow (\texttt{while} loop, \texttt{switch} statement), and method creation. It provides a rudimentary banking simulation, handling user choices and displaying the balance.  Further development would include implementing the deposit and withdrawal functionalities.  The provided text focuses on the initial structure and implementation of the \texttt{showBalance} method.  Error handling for invalid input is included, but detailed input validation is not.


\section{Summary}

This text describes the implementation of a simple banking program in Java.  The author details the creation of two methods: `deposit` and `withdraw`.  Both methods handle user input, validate the amount (checking for negative values and insufficient funds in the `withdraw` method), and update the account balance.  A key improvement is moving the `Scanner` object from a local scope to a static class scope to avoid creating multiple scanner instances.


\section{Deposit Method Implementation}

\subsection{Method Signature and Local Variables}

The `deposit` method is declared as a static method with a double return type. It contains a local variable `amount` of type `double` to store the user's deposit.

\subsection{User Input and Validation}

The method prompts the user to enter the deposit amount using `System.out.print`.  It then uses the class-scoped static `Scanner` object to read the input as a double.  Input validation checks if the amount is negative; if so, an error message is printed, and 0 is returned.

\subsection{Returning the Deposit Amount}

If the amount is valid (non-negative), the method returns the `amount`.  This returned value is then added to the account balance in the main program.

\section{Withdraw Method Implementation}

\subsection{Method Signature and Parameters}

The `withdraw` method is also a static method returning a double. It takes the current balance as a `double` parameter.  A local variable `amount` stores the withdrawal amount.

\subsection{User Input and Validation}

Similar to the `deposit` method, the `withdraw` method prompts the user for an amount and reads it using the class-scoped `Scanner`.  It performs validation:

\begin{itemize}
    \item If the withdrawal amount exceeds the balance, it prints "insufficient funds" and returns 0.
    \item If the amount is negative, it prints "amount can't be negative" and returns 0.
\end{itemize}

\subsection{Returning the Withdrawal Amount}

If the withdrawal is valid, the method returns the `amount`. This returned value is subtracted from the account balance in the main program.


\section{Scanner Management}

The author highlights the importance of managing the `Scanner` object.  Initially declared within the main method, it is refactored to be a static member variable of the class. This improves efficiency by reusing a single scanner instance and preventing potential resource leaks.  This avoids creating multiple `Scanner` instances, which could lead to issues if left open.


\section{Java Banking Program}

This program allows users to deposit, withdraw, and check their balance.  Key features include:

\begin{itemize}
    \item \textbf{Deposit Functionality:} Allows users to deposit funds into their account.
    \item \textbf{Withdrawal Functionality:} Allows users to withdraw funds, with checks for insufficient funds and invalid amounts (negative values).
    \item \textbf{Balance Display:} Shows the current account balance after each transaction.
    \item \textbf{Error Handling:} Handles cases of insufficient funds and attempts to withdraw negative amounts.
    \item \textbf{Exit Message:} Provides a goodbye message upon program exit.
\end{itemize}

\subsection{Example Usage}

The program demonstrates functionality with example transactions, including deposits, withdrawals of various amounts (valid and invalid), and displaying the final balance.  A final exit message is also shown.


\section{Java Dice Rolling Program}

This program simulates rolling multiple dice and displays the result, along with an ASCII art representation (not implemented in the provided text).

\begin{itemize}
    \item \textbf{User Input:} Prompts the user to enter the number of dice to roll.
    \item \textbf{Input Validation:} Checks if the input is greater than zero.  Handles cases where the user enters a non-positive number.
    \item \textbf{Dice Rolling:} Generates random numbers between 1 and 6 (inclusive) for each die.
    \item \textbf{Sum Calculation:} Calculates the total sum of all dice rolls.
    \item \textbf{Output:} Displays each individual die roll and the final sum.
\end{itemize}


\subsection{Program Structure and Steps}

The program follows these steps:

\begin{enumerate}
    \item Declare variables (Scanner, Random, number of dice, total).
    \item Get the number of dice from the user using \texttt{Scanner}.
    \item Check if the number of dice is greater than zero.
    \item If valid, roll all the dice using a \texttt{for} loop and a \texttt{Random} object.
    \item Calculate the total sum of the dice rolls using an augmented assignment operator (\texttt{+=}).
    \item Display the results (individual rolls and the total sum).
\end{enumerate}

\subsection{Code Snippets and Explanations}

The provided text includes explanations and code snippets illustrating variable declarations, user input processing, random number generation, loop iterations, and sum calculations.  The use of  \texttt{Scanner.nextInt()}, \texttt{Random.nextInt()}, and the augmented assignment operator (\texttt{+=}) are highlighted.  Error handling for negative dice rolls is also demonstrated.  Import statements for \texttt{java.util.Scanner} and \texttt{java.util.Random} are mentioned.  The importance of closing the \texttt{Scanner} (\texttt{scanner.close()}) is also noted.

The text describes the use of  \textbackslash{}print and \textbackslash{}println for output.  The concept of breaking down a large project into smaller steps is emphasized for beginner programmers.


\section{Dice Rolling Program in Java}

This program simulates rolling dice and displays ASCII art based on the results.

\subsection{Dice Rolling and Summation}

The program first takes user input for the number of dice to roll. It then uses a \texttt{for} loop to simulate rolling each die (generating random numbers between 1 and 6). The results are added to a \texttt{total} variable.  After the loop, the \texttt{total} is printed.

\subsection{ASCII Art for Dice}

A static method \texttt{printDie(int roll)} is created to display ASCII art representing the dice roll.  This method uses multi-line strings to define the art for each possible die face (1 to 6).  These strings include vertical bars (\texttt{\textbackslash|}) and bullet points (\texttt{\textbullet}).

\subsection{Displaying Dice Results}

An enhanced \texttt{switch} statement is used to select and print the appropriate ASCII art based on the value of each die roll. A default case handles invalid rolls.  The \texttt{printDie} method is called for each die roll before displaying the result.

\subsection{Key Points}

\begin{itemize}
    \item Uses a \texttt{for} loop to simulate multiple dice rolls.
    \item Calculates the sum of the dice rolls.
    \item Employs a static method to generate ASCII art for dice faces.
    \item Uses a \texttt{switch} statement to select the correct ASCII art based on the die roll.
    \item Handles invalid rolls with a default case.
\end{itemize}


\section{Arrays in Java}

This section introduces arrays in Java.

\subsection{Arrays as Collections}

An array is described as a collection of values of the same data type.  It's presented as a more advanced version of a variable that can hold multiple values.

\subsection{Array Declaration and Initialization}

An example shows how to declare and initialize a string array:

\texttt{String[] fruits = \{"apple", "orange", "banana", "coconut"\};}

This creates an array named \texttt{fruits} that can store multiple string values.

\subsection{Accessing Array Elements}

To access individual elements (values) within the array, an index is used.  Remember that indexing starts from 0.  For example, \texttt{fruits[0]} accesses the first element ("apple").

\subsection{Arrays as Reference Data Types}

Printing the array name gives a memory address, as arrays are reference data types.  Each value within the array is referred to as an element.  Accessing these elements requires using the index number.


\section{Introduction to Arrays in Java}

\subsection{Basic Array Operations}

\begin{itemize}
    \item Accessing elements: Arrays are zero-indexed.  Attempting to access a non-existent element throws an \textit{IndexOutOfBoundException}.  For example,  an array with elements at indices 0, 1, 2, 3 will throw an exception if you try to access index 4.
    \item Modifying elements: Elements can be changed by specifying the index using square brackets (e.g.,  \texttt{fruits[0] = "pineapple";}).
    \item Getting array length: The \texttt{.length} property returns the number of elements in the array (e.g., \texttt{fruits.length}).
    \item Iterating through arrays: A \texttt{for} loop can iterate through an array using an index.  An enhanced \texttt{for} loop (or \texttt{for-each} loop) simplifies iteration by directly accessing each element.
\end{itemize}

Example using a \texttt{for} loop:

\begin{verbatim}
for (int i = 0; i < fruits.length; i++) {
    System.out.print(fruits[i] + " ");
}
\end{verbatim}

Example using an enhanced \texttt{for} loop:

\begin{verbatim}
for (String fruit : fruits) {
    System.out.print(fruit + " ");
}
\endverbatim

\subsection{Array Manipulation}

\begin{itemize}
    \item Sorting arrays: The \texttt{java.util.Arrays.sort()} method sorts the array elements alphabetically (for strings) or in ascending order (for numbers).  Requires importing \texttt{java.util.Arrays}.
    \item Filling arrays: The \texttt{java.util.Arrays.fill()} method fills all elements of an array with a specified value. Requires importing \texttt{java.util.Arrays}.
\end{itemize}


\subsection{User Input into Arrays}

Before assigning values to an array, the array size must be specified.  Space for the array is allocated during creation.  For example, to create an array of strings named \texttt{foods}, you would do the following:

\begin{verbatim}
String[] foods = new String[size]; // size needs to be defined beforehand.
\endverbatim

This creates an array named \texttt{foods} which can hold \texttt{size} number of strings.



\section{Summary}

This document provides a basic introduction to arrays in Java, covering element access, modification, iteration using \texttt{for} and enhanced \texttt{for} loops,  length determination, sorting using \texttt{Arrays.sort()}, filling using \texttt{Arrays.fill()}, and handling user input into arrays.  The importance of understanding array indexing and the \texttt{IndexOutOfBoundException} is emphasized.  The use of the \texttt{java.util.Arrays} class is also explained.


\section{Entering User Data into an Array in Java}

This document explains how to populate a Java array with user-supplied data.  The key is to first create an empty array of the appropriate size.  If the size is known beforehand, it can be hardcoded; otherwise, user input is required.


\subsection{Creating and Filling an Array with a Predetermined Size}

Initially, we might try creating an empty array using empty curly braces `{}`. This creates an array with zero elements, which prevents assigning values.  Attempting to do so results in an `ArrayIndexOutOfBoundsException`.

\begin{itemize}
    \item To avoid this, create an empty array with a specified size using the `new` keyword:  `String[] foods = new String[3];` This creates an array capable of holding three elements.
    \item We can then assign values to the array elements using indexing: `foods[0] = "pizza"; foods[1] = "taco"; foods[2] = "hamburger";`
    \item Enhanced for loops simplify iteration:  
    \begin{verbatim}
for (String food : foods) {
    System.out.println(food);
}
    \end{verbatim}
\end{itemize}


\subsection{Creating and Filling an Array with User-Specified Size}

For more flexibility, let's allow the user to specify the array size.

\begin{itemize}
    \item Declare a `Scanner` to get user input: `Scanner scanner = new Scanner(System.in);`  Remember to import `java.util.Scanner` and close the scanner (`scanner.close()`) when finished.
    \item Prompt the user for the desired array size: `System.out.print("What number of food do you want? ");`
    \item Read the size as an integer: `int size = scanner.nextInt();`
    \item Instantiate the array using the user-provided size:  `String[] foods = new String[size];`
    \item Use a `for` loop to iterate and get food items from the user:
    \begin{verbatim}
for (int i = 0; i < foods.length; i++) {
    System.out.print("Enter a food: ");
    foods[i] = scanner.nextLine(); 
}
    \end{verbatim}
    \item Note:  `scanner.nextLine()` is used to consume the newline character left by `scanner.nextInt()`.
    \item Finally, print the filled array.
\end{itemize}

\subsection{Summary}

To populate an array with user-entered values:

\begin{itemize}
    \item Determine the array size.  This may be predetermined or obtained from user input.
    \item Create an empty array of the determined size using `new String[size]`.
    \item Use a loop (e.g., `for` loop) to iterate through the array and populate each element using user input (e.g., via `Scanner`).
\end{itemize}  Remember to handle potential exceptions (e.g., `InputMismatchException` if the user enters non-numeric data for the size) and close the `Scanner` to release system resources.


\section{Searching Through an Array in Java}

\subsection{Linear Search with Integer Array}

This section details how to perform a linear search through an integer array in Java.  Key points include:

\begin{itemize}
\item Creating an integer array (e.g., \texttt{numbers}).
\item Specifying a \texttt{target} integer value to search for.
\item Utilizing a \texttt{for} loop to iterate through the array.
\item Using an \texttt{if} statement to check if the current element matches the \texttt{target}.
\item Printing the index if a match is found, using a \texttt{break} statement to exit the loop after finding the \texttt{target}.
\item Handling the case where the \texttt{target} is not found by introducing a Boolean variable (\texttt{isFound}) and printing a "not found" message if \texttt{isFound} remains \texttt{false} after the loop.
\end{itemize}

The code efficiently searches the array, and if the target number is found, it prints its index. If not, it displays an appropriate "element not found" message.


\subsection{Linear Search with String Array}

Searching a string array requires a slightly different approach due to the nature of strings as reference types. Key differences:

\begin{itemize}
\item The array now contains strings (e.g., \texttt{fruits}).
\item The \texttt{target} is also a string.
\item Instead of using the \texttt{==} operator for comparison, the \texttt{equals()} method of the String class is used to compare string values, not memory addresses.
\end{itemize}


\subsection{User Input and Improved Search}

This section enhances the search functionality by incorporating user input:

\begin{itemize}
\item  Using \texttt{java.util.Scanner} to accept user input for the \texttt{target} string.
\item Prompting the user to enter a fruit to search for.
\item Closing the scanner using \texttt{scanner.close()}.
\end{itemize}

This improvement allows the program to search for a user-specified element dynamically.


\section{Variable Arguments (Varargs) in Java}

This section explains variable arguments (varargs) in Java:

\begin{itemize}
    \item Varargs allow a method to accept a variable number of arguments.
    \item They eliminate the need for multiple overloaded methods with varying parameter counts.  Overloaded methods share the same name but have different parameter lists.
\end{itemize}

The use of varargs simplifies code by enabling a single method to handle different numbers of input parameters.  The example provided showcases a contrast between using multiple overloaded methods and a single method using varargs.

\textbf{Summary:} The text describes how to implement a linear search algorithm in Java for both integer and string arrays, improving it by adding user input.  It also explains the benefits of using variable arguments (varargs) to create more flexible methods, highlighting the difference between using multiple overloaded methods and a single method utilizing varargs.


```latex
\section{Variable Arguments in Java}

\subsection{The \texttt{add} Method}

This section explains how to create a method that accepts a variable number of integer arguments in Java.  Instead of creating multiple overloaded methods, a single method using varargs simplifies the process.

\begin{itemize}
    \item Declare a method with a return type (e.g., \texttt{int}).
    \item Use an ellipsis (\texttt{...}) after the data type of the parameter (e.g., \texttt{int... numbers}).  This tells the Java compiler to pack the arguments into an array.
    \item Inside the method, access the arguments as an array (e.g., \texttt{numbers}).
    \item Iterate through the array (using a \texttt{for} or enhanced \texttt{for} loop) to perform operations (e.g., summation).
    \item Return the result.
\end{itemize}

Example:

\begin{verbatim}
public static int add(int... numbers) {
    int sum = 0;
    for (int number : numbers) {
        sum += number;
    }
    return sum;
}
\end{verbatim}

\subsection{The \texttt{average} Method}

This section demonstrates creating a method to calculate the average of a variable number of \texttt{double} arguments, handling the case of zero arguments to avoid division by zero.

\begin{itemize}
    \item Create a static method with a \texttt{double} return type.
    \item Use varargs with a \texttt{double} data type (e.g., \texttt{double... numbers}).
    \item Include an \texttt{if} statement to check if the array length is zero. If so, return zero to avoid a \texttt{DivisionByZero} exception.
    \item Otherwise, calculate the sum and divide by the array length to find the average.
\end{itemize}

Example:

\begin{verbatim}
public static double average(double... numbers) {
    if (numbers.length == 0) {
        return 0;
    }
    double sum = 0;
    for (double number : numbers) {
        sum += number;
    }
    return sum / numbers.length;
}
\end{verbatim}

\section{Two-Dimensional Arrays in Java}

This section explains two-dimensional arrays (also known as multi-dimensional arrays) in Java.  A 2D array is an array of arrays.  It's useful for representing matrices or tables of data.

\begin{itemize}
    \item Declare a 2D array using double square brackets (e.g., \texttt{String[][] groceries}).
    \item Initialize the array with nested curly braces, where each inner set of braces represents a row (e.g., \texttt{{"apple", "orange", "banana"}, {"potato", "onion", "carrot"}, ...}).
    \item Access elements using nested loops or enhanced \texttt{for} loops.
\end{itemize}

Example:

\begin{verbatim}
String[][] groceries = {
    {"apple", "orange", "banana"},
    {"potato", "onion", "carrot"},
    {"chicken", "pork", "beef", "fish"}
};
\end{verbatim}

This example demonstrates the creation and use of variable arguments and two-dimensional arrays in Java.  Varargs provide a concise way to handle a varying number of arguments, while 2D arrays are efficient for storing and manipulating tabular data.
```


\section{Two-Dimensional Arrays in Java}

\subsection{Grocery Example}

This section explains two-dimensional arrays using a grocery list example.  Each element in the outer array is itself an array (a row).

\begin{itemize}
    \item The outer array's elements (rows) are string arrays.
    \item Each inner array (row) contains strings (e.g., food items).
    \item Accessing elements requires two indices: one for the row, one for the column.
    \item  Example: Replacing "apple" with "pineapple" at row 0, column 0.
    \item  Demonstrates creating and modifying a 2D array.
\end{itemize}

\subsection{Telephone Pad Example}

This example shows a 2D array representing a telephone keypad.

\begin{itemize}
    \item The array is of type \texttt{char}, storing characters in each cell.
    \item The outer array represents rows; each inner array represents a row of characters (numbers and symbols).
    \item Nested loops are used to iterate and print the array, creating the keypad's visual representation.
    \item Illustrates data type considerations for nested loops and element access in a 2D array.
\end{itemize}


\subsection{Quiz Game Project}

This section outlines a Java project to create an interactive quiz game.

\begin{itemize}
    \item Uses a 1D array for questions.
    \item Uses a 2D array for answer options, where each row represents a question and each column represents an answer choice.
    \item The project involves user input and score calculation.
    \item Breaks down the project into steps for easier implementation.
\end{itemize}

\section{Summary}

This document explains the concept of two-dimensional arrays in Java with illustrative examples: a grocery list, a telephone keypad, and a quiz game.  It emphasizes the use of nested loops for iteration, data type considerations, and the methods for accessing and modifying elements within a 2D array using row and column indices.  The examples demonstrate how to create, populate, and manipulate 2D arrays to store and represent data in a tabular format.


\section{Java Quiz Game Project Outline}

This document outlines the creation of a Java quiz game.  The game will present five multiple-choice questions to the user, track their score, and provide a final result.

\subsection{Data Structures}

\begin{itemize}
    \item \textbf{questions}: A String array storing the five quiz questions.  Example questions include:  "What is the main function of a router?", "Which part of the computer is considered the brain?", "What year was Facebook launched?", "Who is known as the father of computers?", and "What was the first programming language?".
    \item \textbf{options}: A two-dimensional String array storing the options for each question. Each inner array contains four possible answers for the corresponding question.
    \item \textbf{answers}: An integer array storing the index (1-4) of the correct answer for each question.
    \item \textbf{score}: An integer variable to track the user's score. Initialized to 0.
    \item \textbf{guess}: An integer variable to store the user's answer.
\end{itemize}

\subsection{Program Logic}

\begin{itemize}
    \item The program begins with a welcome message:  "Welcome to the Java Quiz Game!"
    \item A \texttt{for} loop iterates through the \texttt{questions} array.
    \item Inside the outer loop, each question is displayed.
    \item A nested loop iterates through the corresponding options in the \texttt{options} array and displays them.
    \item The user is prompted to enter their answer (1-4).
    \item The user's answer is compared to the correct answer stored in the \texttt{answers} array.
    \item The \texttt{score} is updated based on the correctness of the answer.
    \item After the loop completes, the final score is displayed.
    \item A \texttt{Scanner} object is used for user input, and it is closed at the end of the program using \texttt{scanner.close()}.
\end{itemize}

\subsection{Code Structure (Summary)}

The code will utilize loops to manage questions and options.  A \texttt{Scanner} will handle user input.  Arrays will store questions, options, and answers.  The program will display a welcome message, present questions, collect user input, check answers, update the score, and finally display the final score.  Error handling (e.g., for invalid user input) is not explicitly detailed in the provided text.  The use of \textbackslash{}java\textbackslash{}util\textbackslash{}Scanner is implied.


\section{Interactive Quiz Game in Java}

This section describes the creation of an interactive quiz game using Java.

\subsection{Game Mechanics}

\begin{itemize}
    \item The game presents questions one at a time.
    \item  Each question has multiple-choice options displayed to the user.
    \item The user inputs their answer (a number).
    \item The game checks if the answer is correct.
    \item Correct answers are rewarded with points.
    \item The final score is displayed at the end.
\end{itemize}

The code uses nested loops: an outer loop iterates through questions, and an inner enhanced for loop iterates through options for each question.  The data type of each value within the options array is a string.  User input is handled using \texttt{Scanner.nextInt()}.  An \texttt{if-else} statement checks the user's guess against the correct answer.  The score is incremented for correct answers and the final score is calculated and displayed using string concatenation and the \texttt{questions.length} property.


\subsection{Code Snippets and Explanations}

The code includes snippets demonstrating:

\begin{itemize}
    \item  Printing each option from a 2D array:  \texttt{options[i]} accesses the inner array.
    \item Getting user input: \texttt{guess = scanner.nextInt()}
    \item Checking the answer: \texttt{if (guess == answers[i])}
    \item Providing feedback: printing "correct" or "wrong" messages.
    \item Calculating and displaying the final score.
\end{itemize}

Example of correct/incorrect feedback:\texttt{if (guess == answers[i]) \{ System.out.println("Correct!"); \} else \{ System.out.println("Wrong!"); \}}

Final score calculation and display: \texttt{System.out.println("Your final score is " + score + " out of " + questions.length);}

\section{Rock Paper Scissors Game in Java}

This section outlines the creation of a Rock Paper Scissors game in Java for beginners.


\subsection{Project Breakdown}

The project is broken down into smaller, manageable steps:

\begin{itemize}
    \item Declare variables.
    \item Get user choice.
    \item Get a random choice for the computer.
    \item Check win conditions.
    \item Ask to play again.
    \item Display a goodbye message.
\end{itemize}

This approach is recommended for beginners to improve organization and understanding.


\subsection{Variable Declarations and Initialization}

The code includes declarations and initializations for:

\begin{itemize}
    \item \texttt{Scanner} object for user input: \texttt{Scanner scanner = new Scanner(System.in);}
    \item \texttt{Random} object for computer's choice: \texttt{Random random = new Random();}
    \item String array for choices: \texttt{String[] choices = \{"Rock", "Paper", "Scissors"\};}
    \item Variables to store player's choice, computer's choice, and whether to play again.
\end{itemize}

Import statements are used for the \texttt{Scanner} and \texttt{Random} classes:  \texttt{import java.util.Scanner;} and \texttt{import java.util.Random;}.  Resource management is addressed by closing the scanner: \texttt{scanner.close();}.


\subsection{Game Logic}

A \texttt{while} loop continues the game as long as the user wants to play again (\texttt{playAgain == "yes"}). The loop incorporates: prompting the user for their move, generating a random choice for the computer, and determining the winner.  The specific logic for determining the winner is not detailed in the provided text.

The text summarizes the creation of two Java games: an interactive quiz and a Rock Paper Scissors game.  Both examples emphasize a structured approach to programming, including breaking down complex tasks into smaller steps and using appropriate data structures and control flow mechanisms.  Emphasis is placed on beginner-friendly explanations and code organization.


\section{Rock, Paper, Scissors Game Logic}

\subsection{User Input and Validation}

We use a scanner to obtain user input.  The program accepts a string using the \texttt{nextLine} method.  Input is converted to lowercase for consistency.

\begin{itemize}
    \item Obtain user input (playerChoice).
    \item Convert playerChoice to lowercase.
    \item Validate playerChoice using an \texttt{if} statement:
    \begin{itemize}
        \item \texttt{!(playerChoice.equals("rock") \&\& playerChoice.equals("paper") \&\& playerChoice.equals("scissors"))}
        \item If the condition is true, output "Invalid choice".
    \end{itemize}
\end{itemize}

\subsection{Computer's Choice}

The computer's choice is randomly generated:

\begin{itemize}
    \item \texttt{computerChoice = choices[random.nextInt(3)]} where \texttt{choices} is an array containing "rock", "paper", and "scissors".
    \item Output the computer's choice.
\end{itemize}


\subsection{Determining the Winner}

The game logic uses a series of \texttt{if-else if-else} statements to determine the outcome:

\begin{itemize}
    \item \textbf{Tie Condition:} \texttt{if (playerChoice.equals(computerChoice))} output "It's a tie".
    \item \textbf{Win Conditions:}  \texttt{else if} statements check all possible win scenarios.  These can be combined into a single \texttt{else if} for better readability. For example:
    \begin{itemize}
        \item Player chooses rock and computer chooses scissors.
        \item Player chooses paper and computer chooses rock.
        \item Player chooses scissors and computer chooses paper.
    \end{itemize}
    \item \textbf{Lose Condition:} An \texttt{else} statement handles the remaining scenarios (losses for the player).  Output "You lose".
\end{itemize}

\subsection{Summary}

This code implements a rock, paper, scissors game. It takes user input, validates it, generates a random computer choice, and then determines the winner based on a set of win conditions using nested if/else if/else statements.  The code prioritizes clarity and readability.  Error handling is included for invalid user input. The use of the \texttt{equals} method for String comparison is highlighted.


\section{Rock Paper Scissors Game in Java}

\subsection{Game Logic and Implementation}

\begin{itemize}
    \item The core game logic involves comparing the user's choice (Rock, Paper, or Scissors) with the computer's randomly selected choice.
    \item Win/lose/tie conditions are evaluated using nested \texttt{if} statements and the logical OR operator (\texttt{\textbackslash{}|}) for readability.  Parentheses are used to group conditions clearly.
    \item A \texttt{do-while} loop allows the user to play multiple rounds until they choose to quit.
    \item User input is handled using a \texttt{Scanner}, converted to lowercase, and validated.  Invalid input causes the current iteration to be skipped using \texttt{continue}.
    \item A final "Thanks for playing" message is displayed upon exiting the loop.
\end{itemize}


\subsection{Code Structure and Improvements}

The code utilizes multiple \texttt{if} statements to manage game logic, which could be potentially streamlined.  The use of a \texttt{do-while} loop for repeated gameplay is efficient.  Error handling is incorporated to manage invalid user input.  Readability could be further enhanced by refactoring the nested \texttt{if} statements.


\section{Java Slot Machine Program}

\subsection{Program Overview}

This program simulates a slot machine game.  The user starts with a balance and bets an amount on each spin.  The result is determined by randomly generated emojis.  Matching symbols result in a payout, updated in the user's balance.  The game continues as long as the user has a balance greater than zero and chooses to play.


\subsection{Implementation Steps}

\begin{itemize}
    \item \textbf{Variable Declaration:}  Variables are declared to store user balance, bet amount, payout, and the array of emojis representing the slot machine's symbols. A \texttt{Scanner} is used for user input.
    \item \textbf{Welcome Message and Input:} A welcome message is displayed, along with the available symbols.  The user is prompted to enter their bet.
    \item \textbf{Bet Validation:}  The program verifies that the bet is within the allowed range (greater than zero and less than or equal to the balance).
    \item \textbf{Slot Spin and Payout:} A method generates and displays a row of random emojis.  Win conditions are evaluated based on matching symbols, and payouts are calculated.
    \item \textbf{Balance Update:} The user's balance is updated based on the bet and payout.
    \item \textbf{Play Again Prompt:}  The user is asked if they want to play again.
    \item \textbf{Exit Message:} If the user chooses not to play again, or if their balance reaches zero, an exit message is displayed.
\end{itemize}

The program uses a \texttt{Scanner} to handle user input and performs input validation to prevent errors.  A clear structure with comments outlining each step improves readability and maintainability.


\section{Java Slots Game Implementation}

This document summarizes the implementation of a Java-based slot machine game.  The game uses emojis as symbols and incorporates user input for betting.

\subsection{Emoji Selection and Game Setup}

The game uses emojis (cherries, watermelon, lemons, bells, stars) as symbols.  These are displayed using a method that is not detailed here.  The game interface is enhanced with asterisks as separators for visual appeal.

\subsection{Game Logic}

\begin{itemize}
    \item \textbf{Welcome Message and Balance Check:} The game starts with a welcome message, displaying the player's current balance.  A \texttt{while} loop continues the game as long as the balance is greater than zero.
    \item \textbf{Bet Placement:} The player is prompted to enter their bet amount.
    \item \textbf{Bet Validation:}
    \begin{itemize}
        \item If the bet exceeds the balance, an "insufficient funds" message is displayed, and the loop continues to the next iteration using the \texttt{continue} keyword.
        \item If the bet is less than or equal to zero, an error message is shown, and the loop continues.
        \item Otherwise, the bet is deducted from the balance using the augmented assignment operator (\texttt{balance -= bet}).
    \end{itemize}
    \item \textbf{Spinning the Reels:}  After a valid bet, a "spinning" message is displayed. A method \texttt{spinRow()} is called to generate a random row of symbols.
\end{itemize}


\subsection{The \texttt{spinRow()} Method}

This static method returns a string array representing the row of symbols.

\begin{itemize}
    \item \textbf{Symbol Array:} An array of strings (\texttt{symbols}) is declared and initialized with the emojis.
    \item \textbf{Row Generation:} An empty string array \texttt{row} (size 3) is created.
    \item \textbf{Random Number Generation:} A \texttt{java.util.Random} object is used to generate random numbers to select symbols from the \texttt{symbols} array.  The index is generated using \texttt{random.nextInt(symbols.length)}.
    \item \textbf{Return Value:} The \texttt{row} array (containing the randomly selected emojis) is returned.
\end{itemize}

The implementation uses a \texttt{java.util.Random} object to select random symbols from the emoji array.  The number of symbols is dynamically handled using the \texttt{length} property of the \texttt{symbols} array, making the code adaptable to different symbol sets.  The use of \texttt{continue} effectively skips iterations under invalid bet conditions.  The code also demonstrates good coding practices such as using augmented assignment operators.  The use of \textbackslash{} is demonstrated implicitly in the LaTeX code itself (e.g., in the \texttt{while} loop description).


\section*{Summary}

This text describes the implementation of a simple slot machine game.  The code generates three random symbols, checks for matching symbols, and calculates a payout based on the matching symbols. The main functions are `spinRow` (generates random symbols), `printRow` (displays the symbols), and `getPayout` (calculates the winnings). The payout is determined by a switch statement based on the matching symbols.

\section*{Code Implementation Details}

\subsection*{`spinRow` Method}
\begin{itemize}
    \item Generates three random symbols from an array of emojis using a \texttt{for} loop.
    \item Returns a string array containing the three symbols.
\end{itemize}

\subsection*{`printRow` Method}
\begin{itemize}
    \item Takes a string array (the row of symbols) as input.
    \item Prints the symbols separated by vertical bars and spaces.  Uses the \texttt{String.join} method for formatting.
\end{itemize}

\subsection*{`getPayout` Method}
\begin{itemize}
    \item Takes the string array (row) and the bet amount as input.
    \item Checks if all three symbols are identical using \texttt{equals} for string comparison.
    \item Uses a \texttt{switch} statement to determine the payout based on the matching symbol:
    \begin{itemize}
        \item Cherry: bet \* 3
        \item Watermelon: bet \* 4
        \item Lemon: bet \* 5
        \item Bell: bet \* 10
        \item Star: bet \* 20
        \item Default (no match): 0
    \end{itemize}
    \item Returns the payout amount.
\end{itemize}

\subsection*{Main Logic}
\begin{itemize}
    \item Calls \texttt{spinRow} to get the symbols.
    \item Calls \texttt{printRow} to display the symbols.
    \item Calls \texttt{getPayout} to calculate the winnings.
    \item If the payout is greater than zero, prints a winning message including the payout amount.
\end{itemize}

The code utilizes string manipulation, loops, conditional statements, and a switch statement to implement the slot machine functionality.  The use of \texttt{String.join} enhances the output formatting.  Careful attention is given to string comparison, using \texttt{equals} instead of \texttt{==} for string equality checks.


\section{Java Slot Machine Game}

\subsection{Game Logic}

\begin{itemize}
    \item Balance calculation: \texttt{balance += ourPayout}
    \item Loss message: "Sorry you lost this round"
    \item Win message:  Displays winnings and updated balance.
    \item Payout based on matching symbols (3 matching symbols for maximum payout, reduced payout for 2 matching symbols).
    \item Bug fix: Corrected an issue in the \texttt{else if} statement checking for matching second and third symbols.  Both symbols should be checked against the same index.
    \item User input for continued play (\texttt{'Y'} or \texttt{'N'}).
    \item Input buffer clearing after integer input using \texttt{scanner.nextLine()}.
    \item Game over message displaying final balance.
\end{itemize}

\subsection{Game Example}

Multiple test runs are described, demonstrating wins and losses, and the final balance calculation.


\section{Object-Oriented Programming in Java}

\subsection{What is an Object?}

An object in programming represents a real-world entity.  It can hold data (attributes) and perform actions (methods).  Examples include a phone, a cup of coffee, or a microphone.  Objects are reference data types stored in the Heap.

\subsection{Classes and Objects}

A class serves as a blueprint for creating objects.  We will create a \texttt{Car} class as an example.

\begin{itemize}
    \item \textbf{Attributes} (data a \texttt{Car} object holds):
    \begin{itemize}
        \item \texttt{make} (String, e.g., "Ford")
        \item \texttt{model} (String, e.g., "Mustang")
    \end{itemize}
    \item Methods (actions a \texttt{Car} object can perform):  These are not explicitly defined in the provided text.
\end{itemize}

The text describes creating a \texttt{Car} class in Java with \texttt{make} and \texttt{model} attributes.  A \texttt{main} class is also mentioned.

The provided text describes a Java slot machine game and introduces the concept of object-oriented programming with a \texttt{Car} class example.  The game logic involves tracking the player's balance, determining payouts based on symbol matches, handling user input, and displaying appropriate messages.  The object-oriented programming section explains the basic concepts of objects, attributes, and methods using a \texttt{Car} object as an example.


\section{Object-Oriented Programming in Java: Cars}

\subsection{Car Attributes and Object Creation}

A car object can have attributes like year (integer), price (double), and engine status (Boolean).  For example:

\begin{itemize}
    \item Year: 2025
    \item Price: \$58,000.99
    \item isRunning: \texttt{true} or \texttt{false}
\end{itemize}

Object creation follows a pattern like:  \texttt{car car = new car;}.  Accessing attributes uses the dot operator (\texttt{.}), e.g., \texttt{car.year}, \texttt{car.price}, \texttt{car.isRunning}.  These attributes can be modified, for example, setting \texttt{car.isRunning = true}.


\subsection{Car Methods}

Objects can perform actions through methods.  A \texttt{car} object might have methods like:

\begin{itemize}
    \item \texttt{start()}: Outputs "You start the engine" and sets \texttt{isRunning} to \texttt{true}.
    \item \texttt{stop()}: Outputs "You stop the engine" and sets \texttt{isRunning} to \texttt{false}.
    \item \texttt{drive()}: Outputs "You drive the " + \texttt{car.model}.
    \item \texttt{break()}: Outputs "You break the " + \texttt{car.model}.
\end{itemize}

These methods can be called using the dot operator: \texttt{car.start()}, \texttt{car.stop()}, etc.


\subsection{Limitations of Identical Objects}

Creating multiple \texttt{car} objects results in identical attributes and methods.  This limits creating unique car instances.


\section{Constructors in Java}

Constructors are special methods used to initialize objects, enabling the creation of unique objects with varying attributes.  A demonstration using a \texttt{Student} class is presented.  This class would have attributes such as name, to allow for unique student objects to be created.


\subsection{Summary}

Objects are entities that hold data (attributes) and perform actions (methods). Constructors allow for the initialization of objects with unique values, addressing the limitations of creating identical objects.


\section{Java Student Class and Constructors}

This text describes creating a Java class representing students, highlighting the use of constructors to initialize object attributes and ensure uniqueness.  Key issues addressed include default attribute values and parameter passing.

\subsection{Initial Class Structure and Limitations}

Initially, a `Student` class was created with attributes like name, age, GPA, and `isEnrolled`.  However, a critical flaw existed: all student objects were initialized with the name "SpongeBob."

\begin{itemize}
    \item Problem: All students had the same name.
    \item Solution: Implementing a constructor to initialize attributes.
\end{itemize}

\subsection{Implementing the Constructor}

A constructor, a special method within the class, was introduced to address the naming issue.  This constructor takes the name, age, and GPA as arguments, allowing for unique student object creation.

\begin{itemize}
    \item Constructor syntax:  The constructor's name matches the class name (`Student`), and it takes parameters to initialize the object's attributes.
    \item Parameter passing:  Arguments are passed to the constructor when creating a new `Student` object (using the `new` keyword).
    \item `this` keyword:  The `this` keyword is used within the constructor to assign parameter values to the object's attributes (e.g., `this.name = name`).
\end{itemize}

\subsection{Constructor Parameter Names and Attribute Assignment}

The text emphasizes that constructor parameter names do not need to match attribute names. Although it's recommended to keep them the same for readability, it demonstrates how assigning different parameter names (e.g., `a`, `b`, `c`) still correctly initializes the object's attributes using `this`.

\subsection{Implicit Attribute Initialization}

The `isEnrolled` attribute was implicitly set to `true` within the constructor, highlighting that not all attributes necessarily require explicit parameter passing during object creation.


\subsection{Example Usage and Output}

Multiple `Student` objects were created using the constructor with different names, ages, and GPAs. The code then demonstrates printing the attributes of these objects, showcasing the successful use of the constructor to create unique student instances.  A `study()` method was also briefly mentioned to illustrate further class functionality.

\begin{itemize}
    \item Multiple student objects created (SpongeBob, Patrick, Sandy).
    \item Attributes (name, age, GPA, isEnrolled) printed for each student.
    \item Demonstration of a simple class method (`study()`).
\end{itemize}

The overall text demonstrates fundamental object-oriented programming concepts in Java, focusing on the use of constructors for proper object initialization and the significance of the `this` keyword.


\section{Constructors and Object Initialization in Java}

\subsection{Constructors: A Special Method}

\begin{itemize}
    \item Constructors are special methods within a class used to initialize objects.
    \item They are automatically called when a new object is created.
    \item They can accept arguments to set initial values for object attributes.
    \item Using constructors, objects can be created with unique values.
\end{itemize}

\subsection{Overloaded Constructors}

\begin{itemize}
    \item Overloaded constructors allow a class to have multiple constructors with different parameter lists.
    \item This enables an object to be initialized in various ways, depending on the provided arguments.
    \item Similar to method overloading, constructors with the same name but different parameters are allowed.  
    \item  Useful when some fields are optional during object creation.
\end{itemize}

\textbf{Example: User Class with Overloaded Constructors}

A \texttt{User} class with attributes \texttt{username} (String), \texttt{email} (String), and \texttt{age} (integer) is created.  Overloaded constructors are implemented to handle different scenarios:

\begin{itemize}
    \item Constructor with only \texttt{username}: Sets default values for \texttt{email} and \texttt{age}.
    \item Constructor with \texttt{username} and \texttt{email}: Sets a default value for \texttt{age}.
    \item Constructor with \texttt{username}, \texttt{email}, and \texttt{age}: Sets all attributes.
    \item Constructor with no arguments: Sets default values for all attributes (e.g., "guest" profile).
\end{itemize}

This allows creating \texttt{User} objects with varying amounts of information.


\section{Arrays of Objects in Java}

\subsection{Creating an Array of Objects}

The text describes creating an array of objects in Java. An example using a \texttt{Car} class with \texttt{model} and \texttt{color} attributes is mentioned but the details of the array creation are not fully provided in the excerpt.


\textbf{Summary:} The text explains constructors in Java, specifically focusing on overloaded constructors and their use in creating objects with varying sets of attributes. It also briefly introduces the concept of creating arrays of objects, but lacks the complete code example for the array implementation.  The examples use illustrative scenarios involving SpongeBob, Patrick, and Sandy to demonstrate the concept of overloaded constructors.


\section{Car Class and Array of Objects}

This section demonstrates creating a `Car` class, instantiating multiple `Car` objects, and manipulating them within an array.

\subsection{Car Class}

A `Car` class is defined with attributes for \texttt{model} and \texttt{color}, and a \texttt{drive} method.  The \texttt{drive} method prints a message indicating the car's color and model.

\subsection{Creating Car Objects}

Three `Car` objects (\texttt{car1}, \texttt{car2}, \texttt{car3}) are created with different models and colors:

\begin{itemize}
    \item \texttt{car1}: Red Mustang
    \item \texttt{car2}: Blue Corvette
    \item \texttt{car3}: Yellow Charger
\end{itemize}

\subsection{Array of Car Objects}

An array named \texttt{cars} is created to store the three `Car` objects.  The code demonstrates two ways to populate the array:

\begin{itemize}
    \item  Directly assigning the objects: \texttt{cars = \{car1, car2, car3\}}
    \item Using a for loop to iterate and call the \texttt{drive} method of each car.
    \item Using an enhanced for loop to iterate and call the \texttt{drive} method of each car.
\end{itemize}

The code also shows how to create an array of anonymous objects, directly instantiating the `Car` objects within the array declaration without assigning them to separate variables.

\subsection{Modifying and Accessing Array Elements}

The code demonstrates modifying the color attribute of each car object in the array using an enhanced for loop and then re-iterating to show the updated information.


\section{Static Keyword in Java}

This section explains the use of the \texttt{static} keyword in Java.

\subsection{Static Keyword}

The \texttt{static} keyword is used to create variables or methods that belong to the class itself, rather than to specific instances (objects) of the class.  This is often used for utility methods or shared resources.


\subsection{Demonstration: Friend Class}

A \texttt{Friend} class is created with a \texttt{name} attribute and a constructor.  A \texttt{static} integer variable \texttt{numOfFriends} is used to keep track of the number of \texttt{Friend} objects created.  Each time a new \texttt{Friend} object is created, \texttt{numOfFriends} is incremented.  This demonstrates how a static variable is shared among all instances of the class.


**Summary:** The text explains how to create and use arrays of objects in Java, using the `Car` class as an example.  It then introduces the concept of the `static` keyword, illustrating its use with a `Friend` class to track the number of objects created.  Both examples utilize various loop structures for iteration and manipulation of array elements.


\section{Static Keyword in Java}

\subsection{Illustrative Example: The \texttt{Friend} Class}

We'll create a \texttt{Friend} class to illustrate the use of the \texttt{static} keyword. Initially, each \texttt{Friend} object has its own \texttt{numberOfFriends} variable.  Creating multiple \texttt{Friend} objects results in each object having its own count, even though we intend to track the total number of friends.

\begin{itemize}
    \item Creating two \texttt{Friend} objects, Patrick and another, shows that each object has its own independent count of friends.
    \item The output shows that both friends have one friend each (themselves), rather than two total friends.
\end{itemize}

To track the total number of friends, we use the \texttt{static} modifier:

\begin{itemize}
    \item \texttt{static int numberOfFriends;} declares a static variable shared across all \texttt{Friend} objects.
    \item Now, creating multiple \texttt{Friend} objects correctly increments the single shared \texttt{numberOfFriends} variable.
\end{itemize}

Accessing static members is best done through the class name (e.g., \texttt{Friend.numberOfFriends}) rather than through an object instance (e.g., \texttt{friend1.numberOfFriends}).

Static methods are similarly accessed via the class name:

\begin{itemize}
    \item We create a \texttt{static void showFriends()} method to display the total number of friends.  This method directly accesses the static \texttt{numberOfFriends} variable without using \texttt{this}.
    \item  Calling \texttt{Friend.showFriends()} correctly displays the total count.
\end{itemize}

The \texttt{Math} class provides further examples of static methods (e.g., \texttt{Math.round()}), which are utility methods accessed directly through the class name without object instantiation.

\subsection{Summary of Static Keyword}

The \texttt{static} keyword in Java modifies a variable or method, making it belong to the class itself rather than any specific object.  This is useful for utility methods or shared resources among objects.  Static members are accessed using the class name, not an object instance.


\section{Inheritance in Java}

Inheritance allows a class (child class) to inherit attributes and methods from another class (parent class).  This is analogous to how children inherit traits from parents.

Example: We create an \texttt{Animal} class with attributes and methods common to all animals.  Other classes (e.g., \texttt{Dog}, \texttt{Cat}) can inherit from \texttt{Animal}, inheriting the shared attributes and methods and adding their own specific attributes and methods.  This promotes code reusability and organization.  The provided text only starts the \texttt{Animal} class definition.


\section{Inheritance in Java}

This document summarizes the concept of inheritance in Java using examples of animal, dog, cat, and plant classes.

\subsection{Basic Inheritance}

The core concept is that a child class (e.g., `Dog`, `Cat`) inherits attributes and methods from a parent class (e.g., `Animal`).  This is achieved using the `extends` keyword.

\begin{itemize}
    \item An `Animal` class is created with methods like `eat()`.
    \item `Dog` and `Cat` classes extend `Animal`, inheriting `eat()` without explicitly defining it in `Dog` or `Cat`.
    \item Objects of `Dog` and `Cat` can use the inherited `eat()` method.
    \item This avoids code duplication, adhering to the DRY (Don't Repeat Yourself) principle.
\end{itemize}

\subsection{Extending Inheritance}

Child classes can have their own unique attributes and methods:

\begin{itemize}
    \item `Dog` has a `lives` attribute and a `speak()` method that outputs "woof".
    \item `Cat` has a `lives` attribute and a `speak()` method that outputs "meow".
\end{itemize}


\subsection{Multi-level Inheritance}

Inheritance can span multiple levels:

\begin{itemize}
    \item An `Organism` class is introduced as a grandparent class.
    \item `Animal` extends `Organism`, inheriting an `isAlive` attribute.
    \item `Dog` and `Cat` inherit from `Animal`, thus indirectly inheriting from `Organism`.
    \item A `Plant` class also extends `Organism`, demonstrating that unrelated classes can share a common ancestor.
\end{itemize}

The `isAlive` attribute is inherited by both `Dog`, `Cat`, and `Plant` from `Organism`.  However, methods specific to `Animal` (like `eat()`) are only inherited by `Dog` and `Cat`, not `Plant`, illustrating that inheritance follows the class hierarchy.  The `photosynthesize()` method is unique to the `Plant` class.  A `Dog` object would not have access to `photosynthesize()`.

\subsection{Summary}

Inheritance in Java allows for code reusability and maintainability.  Child classes inherit attributes and methods from parent classes, promoting a clear hierarchical structure and avoiding redundant code.  Multi-level inheritance is possible, creating complex yet organized relationships between classes.  The `extends` keyword is crucial for establishing these relationships.  The example showcases how unrelated classes can share a common ancestor without inheriting methods specific to other branches of the hierarchy.  This demonstrates a clear understanding of the concept of inheritance and how it works within a structured program.


\section{Inheritance and the \texttt{super} Keyword in Java}

This document summarizes the use of the \texttt{super} keyword in Java inheritance, focusing on its role in constructors.  The \texttt{super} keyword refers to the parent (super) class.

\subsection{The \texttt{Person} Class}

We begin by creating a \texttt{Person} class, serving as the parent (super) class.  This class has the following attributes:

\begin{itemize}
    \item \texttt{String first} (first name)
    \item \texttt{String last} (last name)
\end{itemize}

Its constructor takes \texttt{first} and \texttt{last} as arguments and initializes the attributes accordingly. It also includes a \texttt{showName()} method to display the person's full name.


\subsection{The \texttt{Student} Class}

The \texttt{Student} class inherits from the \texttt{Person} class.  It adds the following attribute:

\begin{itemize}
    \item \texttt{double GPA} (grade point average)
\end{itemize}

Because the \texttt{Person} class constructor requires a first and last name, the \texttt{Student} constructor must use the \texttt{super} keyword to pass these arguments to the parent constructor:

\texttt{super(first, last);}

The \texttt{Student} class also includes a \texttt{showGPA()} method to display the student's GPA.


\subsection{The \texttt{Employee} Class}

The \texttt{Employee} class also inherits from the \texttt{Person} class. It adds the following attribute:

\begin{itemize}
    \item \texttt{int salary}
\end{itemize}

Similar to the \texttt{Student} class, the \texttt{Employee} constructor uses the \texttt{super} keyword to call the \texttt{Person} constructor, passing the \texttt{first} and \texttt{last} name arguments.


\subsection{Summary}

The \texttt{super} keyword is crucial when a child class inherits from a parent class with a non-parameterless constructor.  It allows the child class constructor to call the parent class constructor, passing necessary arguments to initialize the parent class's attributes.  This ensures proper object initialization and avoids compilation errors.  Failing to use \texttt{super} appropriately results in compilation errors indicating the absence of a parameterless constructor in the parent class.  Replacing \texttt{super} with the parent class name conceptually illustrates its function.


\section{Employee Class and the \texttt{super} Keyword}

We'll create an \texttt{Employee} class that uses the \texttt{super} keyword to call the constructor of its parent class.  The parent class constructor requires a first and last name as arguments.

\subsection{Constructor and \texttt{ShowSalary} Method}

Within the \texttt{Employee} class, we define a \texttt{ShowSalary} method to display employee information.

\begin{itemize}
    \item We use the \texttt{super} keyword to call the parent class constructor, passing in the first and last name.
    \item The \texttt{ShowSalary} method outputs the employee's name and salary.
\end{itemize}

Example:

\texttt{employee employee = new employee("Rubius", "Hagrid", 50000);}

\texttt{employee.ShowSalary();} //Output: Rubius Hagrid's salary is \$50,000


\subsection{Summary}

The \texttt{super} keyword calls the parent class's constructor.  Child classes must call the parent's constructor if the parent requires arguments.  If the parent doesn't require arguments, calling the parent's constructor is not necessary.


\section{Method Overriding}

Method overriding allows a subclass to provide its own implementation of a method inherited from its parent class. This promotes code reusability and allows for specific implementations in subclasses.

\subsection{Demonstration with Animal, Dog, Cat, and Fish Classes}

We create classes \texttt{Animal}, \texttt{Dog}, \texttt{Cat}, and \texttt{Fish}, where \texttt{Dog}, \texttt{Cat}, and \texttt{Fish} extend \texttt{Animal}. The \texttt{Animal} class has a \texttt{move()} method.

\begin{itemize}
    \item  The \texttt{move()} method in the \texttt{Animal} class is inherited by \texttt{Dog}, \texttt{Cat}, and \texttt{Fish}.
    \item Initially, all three subclasses inherit the same \texttt{move()} method implementation.
    \item We override the \texttt{move()} method in the \texttt{Fish} class to provide a specific implementation for fish ("This animal is swimming").
    \item The \texttt{@Override} annotation is used to indicate that we are overriding a method, which helps prevent errors.
\end{itemize}


\subsection{Output and Importance of \texttt{@Override}}

The output demonstrates that the \texttt{Dog} and \texttt{Cat} objects use the inherited \texttt{move()} method, while the \texttt{Fish} object uses the overridden method. The \texttt{@Override} annotation provides a check to ensure the method is correctly overridden and helps identify potential errors like misspellings.


\section{\texttt{toString()} Method}

The \texttt{toString()} method is inherited from the \texttt{Object} class. It returns a string representation of an object.  The default behavior returns a hash code, but it can be overridden to provide more meaningful details.  This is another example of method overriding.

\subsection{Example with a \texttt{Car} Class}

We create a \texttt{Car} class to demonstrate overriding the \texttt{toString()} method.  The overridden method will return a more descriptive string representation of a \texttt{Car} object.  (The provided text cuts off before the details of the Car class are given.)


\section{Creating and Using a Car Object in Java}

\subsection{Car Class and Constructor}

The code defines a `Car` class with attributes for make, model, year, and color.  A constructor initializes these attributes.  Key points:

\begin{itemize}
    \item The \texttt{Car} class has attributes: \texttt{make} (string), \texttt{model} (string), \texttt{year} (int), \texttt{color} (string).
    \item A constructor initializes these attributes.
    \item  Directly printing a \texttt{Car} object yields a hash code, not the object's details.
\end{itemize}

\subsection{Overriding the \texttt{toString()} Method}

To print meaningful details of the \texttt{Car} object, the \texttt{toString()} method is overridden. This method returns a string containing the car's details. Key points:

\begin{itemize}
    \item The \texttt{@Override} annotation is used.
    \item The method is declared \texttt{public} and has a return type of \texttt{string}.
    \item The method uses \texttt{this} to access the object's attributes.
    \item Overriding \texttt{toString()} allows printing the object's details instead of a hash code.
\end{itemize}

Examples of creating and printing \texttt{Car} objects are shown.


\section{Abstraction in Java}

\subsection{Concept of Abstraction}

Abstraction hides implementation details and shows only essential features.  The analogy of teaching someone to drive (without explaining the engine) is used to illustrate the concept. Key points:

\begin{itemize}
    \item Abstraction hides implementation details.
    \item Users interact with essential features only.
    \item The \texttt{abstract} keyword creates abstract classes and methods.
\end{itemize}

\subsection{Example: Shape Classes}

An example using an abstract \texttt{Shape} class and its subclasses (\texttt{Circle}, \texttt{Triangle}, \texttt{Rectangle}) demonstrates abstraction.  Key points:

\begin{itemize}
    \item An abstract \texttt{Shape} class cannot be instantiated directly.
    \item Subclasses (\texttt{Circle}, \texttt{Triangle}, \texttt{Rectangle}) extend the \texttt{Shape} class.
    \item Abstract classes can contain abstract methods (which must be implemented by subclasses) and concrete methods.
\end{itemize}

The example shows that attempting to create a \texttt{Shape} object directly results in an error.


\textbf{Summary:} The text explains object creation and usage in Java, focusing on overriding the \texttt{toString()} method for meaningful output and introduces the concept of abstraction using an example with geometric shapes.  The \texttt{abstract} keyword and its implications are discussed.


\section{Abstract Classes and Methods in Java}

\subsection{Abstract Classes}

The message shape is abstract; it cannot be instantiated since shape is an abstract class. We can't create any objects from this class.  This adds a little bit of security to our program; we don't want anybody creating any generic shape objects. We want them to create a certain kind of shape: a circle, a triangle, or a rectangle.

\begin{itemize}
    \item We can create \texttt{Circle}, \texttt{Triangle}, and \texttt{Rectangle} objects, but not \texttt{Shape} because it's an abstract class.
    \item Abstract classes can contain abstract methods and concrete methods.
    \item An abstract method is a method that \textit{must} be implemented by its children.
\end{itemize}

In our \texttt{Shape} class, we define an abstract method \texttt{area} using the \texttt{abstract} keyword:

\texttt{abstract double area();}

Every child class of the \texttt{Shape} class needs an \texttt{area} method.  If a child class (like \texttt{Circle}) doesn't implement \texttt{area}, a compiler error occurs.  Implementing this method ensures consistency among child classes.

\subsection{Concrete Methods in Abstract Classes}

Abstract classes can also contain concrete methods.  Concrete methods are defined and inherited. Let's define a \texttt{display} method in the \texttt{Shape} class:

\texttt{void display() \{\textbackslash{} System.out.println("This is a shape"); \textbackslash{}\}}

This method is inherited by \texttt{Circle}, \texttt{Triangle}, and \texttt{Rectangle} classes without needing to be overridden.

\subsection{Implementing Abstract Methods}

In our \texttt{Circle}, \texttt{Triangle}, and \texttt{Rectangle} classes, we must implement the abstract \texttt{area} method. For example, in \texttt{Circle}:

\begin{itemize}
    \item We need a \texttt{radius} attribute.
    \item The constructor takes a \texttt{radius} as input.
    \item The \texttt{area} method calculates the area using the formula:  \texttt{Math.PI * radius * radius}.
\end{itemize}

Similar implementations are needed for \texttt{Triangle} (using base and height) and \texttt{Rectangle} (using length and width).

\subsection{Abstraction Summary}

Abstraction is the process of hiding implementation details from a user and showing only the essential features. The \texttt{Shape} class exemplifies this; we can't directly create \texttt{Shape} objects, forcing the user to create specific shapes (circle, triangle, rectangle). An abstract class can contain abstract methods (which children must implement) or concrete methods (which children inherit).


\section{Interfaces in Java}

An interface is similar to an abstract class, but with key differences.  An interface is a blueprint for a class, specifying a set of abstract methods that any implementing class must define.  Interfaces allow for multiple inheritance, unlike abstract classes.


This LaTeX code summarizes and formats the provided text using sections, subsections, and itemize environments.  It also correctly handles backslashes.  The code focuses on the key concepts of abstract classes and interfaces in Java.


\section{Interfaces in Java: Multiple Inheritance}

This document explains the use of interfaces in Java to achieve multiple inheritance.  The example uses interfaces to define behaviors for prey and predator animals.

\subsection{Creating Interfaces}

We create two interfaces: `Prey` and `Predator`.  The `Prey` interface declares a `flee()` method (void return type), and the `Predator` interface declares a `hunt()` method (void return type).  These methods are not defined within the interfaces; they are only declared.

\subsection{Implementing Interfaces}

\begin{itemize}
    \item \textbf{Rabbit Class:} Implements the `Prey` interface.  It must define the `flee()` method, fulfilling the contract defined by the interface.  The implementation outputs "The rabbit is running away".
    \item \textbf{Hawk Class:} Implements the `Predator` interface. It must define the `hunt()` method, fulfilling the interface's contract. The implementation outputs "The hawk is hunting".
    \item \textbf{Fish Class:} Implements \textit{both} the `Prey` and `Predator` interfaces. It defines both `flee()` (outputs "The fish is swimming away") and `hunt()` (outputs "The fish is hunting") methods. This demonstrates the key difference between interfaces and inheritance: a class can implement multiple interfaces.
\end{itemize}

This example showcases that a class can inherit from multiple interfaces, unlike traditional inheritance where a class can only have one parent class.

\subsection{Summary of Interfaces}

An interface acts as a blueprint, specifying a set of abstract methods that implementing classes must define. It supports multiple inheritance.  There's significant overlap in the use cases of abstract classes and interfaces; familiarity with both is recommended.


\section{Polymorphism in Java}

Polymorphism, meaning "many shapes," is a fundamental concept where objects can be treated as objects of a common superclass.  A dog, for instance, is a dog, but also an animal, an organism, and an object.

\subsection{Vehicle Example}

An abstract `Vehicle` class is created with an abstract `go()` method.  Concrete classes (`Car`, `Bike`, `Boat`) extend the `Vehicle` class and provide their own implementations of the `go()` method. This demonstrates polymorphism because each vehicle type can be treated as a `Vehicle`, but each has its unique behavior defined by its `go()` implementation.  (Note: The provided text cuts off before completing the example of the `Car`, `Bike`, and `Boat` classes and their implementations of the `go()` method).

This example showcases polymorphism in Java and illustrates how objects of different types can be treated as objects of a common superclass, allowing for flexible and extensible code.


\section{Polymorphism in Java: Using Inheritance}

This section explains polymorphism in Java using inheritance, illustrating how objects of different classes can be treated as objects of a common superclass.

\subsection{Vehicle Example}

We create classes \texttt{Car}, \texttt{Bike}, and \texttt{Boat}, all extending a \texttt{Vehicle} class. Each class has a \texttt{go()} method with a specific implementation (e.g., "you drive the car").  Initially, attempting to create an array of \texttt{Car} objects and adding \texttt{Bike} and \texttt{Boat} objects results in errors due to incompatible types.  However, by creating an array of \texttt{Vehicle} objects, we can store instances of all three classes. A \texttt{for} each loop iterates through this array, calling the \texttt{go()} method on each \texttt{Vehicle} object, demonstrating polymorphism.  The output correctly reflects the specific action for each vehicle type.

\begin{itemize}
\item Classes \texttt{Car}, \texttt{Bike}, and \texttt{Boat} extend \texttt{Vehicle}.
\item Each class has a \texttt{go()} method.
\item An array of \texttt{Vehicle} objects can hold instances of all three classes.
\item A \texttt{for} each loop calls \texttt{go()} on each object, showcasing polymorphism.
\end{itemize}

\subsection{Polymorphism with Interfaces}

The example is repeated using interfaces instead of inheritance.  A \texttt{Vehicle} interface is created with a \texttt{go()} method. Classes \texttt{Car}, \texttt{Bike}, and \texttt{Boat} implement this interface. The same array and \texttt{for} each loop approach works, demonstrating that polymorphism can be achieved through interfaces as well.

\begin{itemize}
\item A \texttt{Vehicle} interface is defined with a \texttt{go()} method.
\item Classes implement the \texttt{Vehicle} interface.
\item Polymorphism is achieved through interface implementation.
\end{itemize}

\section{Runtime Polymorphism (Dynamic Polymorphism)}

This section explains runtime polymorphism, where the method executed is determined at runtime based on the object's actual type.  An example using \texttt{Animal}, \texttt{Dog}, and \texttt{Cat} classes is presented.  \texttt{Animal} is an abstract class with an abstract \texttt{speak()} method. \texttt{Dog} and \texttt{Cat} extend \texttt{Animal} and override \texttt{speak()} with their specific implementations.  A user's choice (dog or cat) determines which object is created, and the corresponding \texttt{speak()} method is called at runtime.

\begin{itemize}
    \item \texttt{Animal} is an abstract class with an abstract \texttt{speak()} method.
    \item \texttt{Dog} and \texttt{Cat} extend \texttt{Animal} and override \texttt{speak()}.
    \item User input determines the object type at runtime.
    \item The correct \texttt{speak()} method is called based on the runtime object type.
\end{itemize}

\textbf{Summary:} The document explains polymorphism in Java using both inheritance and interfaces. It highlights how objects of different classes can share a common interface or superclass, allowing for flexible and dynamic method calling based on the object's actual type at runtime.  This is demonstrated through examples involving vehicles and animals.


\section{Runtime Polymorphism in Java}

\subsection{Example: Dog or Cat}

The program will create either a \texttt{Dog} or \texttt{Cat} object based on user input.

\begin{itemize}
    \item User input is obtained using \texttt{java.util.Scanner}.
    \item A prompt asks the user to choose between a dog (1) or a cat (2).
    \item Based on the choice, an \texttt{animal} object is instantiated as either a \texttt{new Dog()} or \texttt{new Cat()}.
    \item The \texttt{speak()} method is called on the \texttt{animal} object.  This demonstrates runtime polymorphism because the specific \texttt{speak()} method (either \texttt{Dog.speak()} or \texttt{Cat.speak()}) is determined at runtime.
\end{itemize}

The program demonstrates runtime polymorphism because the method called (\texttt{speak()}) is determined at runtime based on the object's type.


\section{Getter and Setter Methods in Java}

\subsection{Protecting Object Data}

Getter and setter methods help protect object data by adding rules for accessing and modifying data.  This example uses a \texttt{Car} class.

\begin{itemize}
    \item The \texttt{Car} class has attributes \texttt{model} (String), \texttt{color} (String), and \texttt{price} (int).
    \item Initially, these attributes are public.  Direct access allows easy modification, potentially leading to unwanted changes.
    \item Making the attributes \texttt{private} prevents direct external access.
    \item Getter methods (\texttt{getModel()}, \texttt{getColor()}, \texttt{getPrice()}) provide controlled read access.
    \item Setter methods (not explicitly shown in the text but implied) would provide controlled write access.
\end{itemize}

By using getter and setter methods, we can control how the object's data is accessed and modified, adding a layer of protection and potentially including additional logic within the getter/setter methods (e.g., formatting the price with currency).  The example shows how to use getter methods to access the private attributes of the \texttt{Car} object safely and even modify the returned value of \texttt{getPrice()} to include a currency symbol.


```latex
\section{Getter and Setter Methods in Java}

This demonstration illustrates getter and setter methods in Java.  Setter methods make a field writable.  Consider a `Car` class with attributes: `model`, `color`, and `price`.  We might only want `color` and `price` to be writable, not `model`.

\subsection{Setter Methods}

We create `setter` methods for `color` and `price`, but not for `model`:

\begin{itemize}
    \item \texttt{void setColor(String color)}: \texttt{this.color = color;}
    \item \texttt{void setPrice(int price)}: \texttt{this.price = price;}
\end{itemize}

Attempting to directly access private attributes (e.g., \texttt{car.color = "blue"}) will fail. Instead, we use the setter methods: \texttt{car.setColor("blue"); car.setPrice(5000);}.  Using getter methods, we can then print the updated attributes.  If we wanted to make \texttt{model} writable, we would add a \texttt{setModel} method.  Alternatively, using the \texttt{final} keyword when declaring a field adds an extra layer of security, preventing modification.  Additional logic can be added to setters, such as input validation (e.g., ensuring price is not negative).

\subsection{Price Validation}

For the price, we can add a check:

\begin{verbatim}
if (price < 0) {
    System.out.println("Price can't be less than zero");
} else {
    this.price = price;
}
\end{verbatim}

This prevents setting a negative price.


\section{Aggregation in Programming}

Aggregation represents a "has-a" relationship between objects. One object contains another, but the contained objects can exist independently.  For example, a library "has" books, but the books can exist independently of the library.

\subsection{Book and Library Classes}

We'll create a \texttt{Book} class with attributes \texttt{title} (String) and \texttt{pages} (integer), and a \texttt{Library} class to contain \texttt{Book} objects.

\begin{itemize}
    \item \textbf{Book Class:}  A constructor sets the \texttt{title} and \texttt{pages}. A \texttt{displayInfo()} method returns the book's title and page count.
    \item \textbf{Library Class:} (Not fully detailed in the provided text) This class would contain a collection of \texttt{Book} objects (e.g., using an \texttt{ArrayList}).
\end{itemize}

The example then demonstrates creating several \texttt{Book} objects and (implicitly) adding them to a \texttt{Library} object, illustrating the aggregation relationship.  The books and the library can exist separately, demonstrating the core concept of aggregation.
```


\section{Aggregation in Java}

This section explains aggregation in Java using the example of a library containing books.

\subsection{Book Objects}

The example starts by creating individual book objects with attributes like title and number of pages.  Book objects are created for "The Fellowship of the Ring" (423 pages), "The Two Towers" (352 pages), and "The Return of the King" (416 pages). These objects are then stored in an array named \texttt{books}.  A \texttt{displayInfo} method is implemented to print the details of each book.

\subsection{Library Object}

A \texttt{Library} class is created to represent a library.  A library object has attributes: a name (string), year established (integer), and an array of \texttt{Book} objects.  The \texttt{Library} class also has a \texttt{displayInfo} method which prints the library's name and year, and then iterates through its array of \texttt{Book} objects, calling each book's \texttt{displayInfo} method.

\subsection{Aggregation}

The relationship between the \texttt{Library} and \texttt{Book} objects is aggregation. This means the \texttt{Library} \textit{has-a} relationship with \texttt{Book} objects.  Crucially, the \texttt{Book} objects can exist independently of the \texttt{Library} object.  Deleting the \texttt{Library} class doesn't affect the existence of the \texttt{Book} objects.

\begin{itemize}
    \item Book objects are created independently.
    \item Library object contains an array of book objects.
    \item Deleting the Library object does not delete the Book objects.
\end{itemize}


\section{Composition in Java}

This section explains composition in Java, using the example of a car and its engine.

\subsection{Car and Engine Objects}

Composition represents a \textit{part-of} relationship.  A car is composed of an engine; the engine is a part of the car.  In this example, a \texttt{Car} class is created with attributes such as model (string), year (integer), and an \texttt{Engine} object. The \texttt{Car} object does \textbf{not} contain an array of \texttt{Engine} objects, only one \texttt{Engine} object.


\subsection{Key Difference: Composition vs. Aggregation}

The key difference between composition and aggregation lies in the lifecycle of the objects. In composition, if the main object (e.g., the car) is deleted, the parts (e.g., the engine) are also deleted because they cannot exist independently.  This is not the case in aggregation.

\begin{itemize}
    \item In composition, the part objects cannot exist without the whole object.
    \item In aggregation, the part objects can exist independently of the whole object.
\end{itemize}

The provided text details examples of both aggregation and composition in Java, highlighting their differences through illustrative examples of libraries and books, and cars and engines, respectively.  The primary focus is on the relationship between objects and how their lifecycles influence the type of object relationship (aggregation or composition).


\section{Composition in Java}

This section demonstrates composition in Java using a car and engine example.  The key idea is that an engine is a \textit{part} of a car.

\subsection{Car and Engine Classes}

We create a \texttt{Car} class that uses an \texttt{Engine} object. The \texttt{Engine} class has a \texttt{type} (e.g., "V8").  The \texttt{Car} class constructor takes the model, year, and engine type as arguments.  Argument names can differ from parameter names.  The engine is constructed within the \texttt{Car} constructor.

\begin{itemize}
    \item The \texttt{Engine} class has a constructor taking the engine type as a string.
    \item The \texttt{Car} class constructor takes the model (string), year (int), and engine type (string) as arguments. It creates a new \texttt{Engine} object using the provided engine type.
    \item The \texttt{Car} class has a \texttt{start()} method which calls the \texttt{start()} method of its \texttt{Engine} object and then outputs a message indicating the car is running.
    \item The \texttt{Engine} class has a \texttt{start()} method which outputs a message indicating the engine is starting.
\end{itemize}

\subsection{Demonstration}

A \texttt{Car} object is created with a Corvette model, year 2025, and a V8 engine.  The output shows the car's attributes, including the engine type accessed via the engine object.  Deleting the \texttt{Car} object also deletes the \texttt{Engine} object because of the composition relationship.

\begin{itemize}
    \item Composition implies a "part-of" relationship.  If the main object is deleted, its composed objects are also deleted.
    \item This allows for complex objects to be built from simpler ones.
\end{itemize}


\section{Wrapper Classes in Java}

Wrapper classes allow primitive data types (like \texttt{int}, \texttt{char}, \texttt{double}, \texttt{boolean}) to be treated as objects.  This is particularly useful when working with collections that require objects.

\subsection{Creating Wrapper Objects (Deprecated Method)}

While deprecated since Java 9, creating wrapper objects manually helps visualize the concept.  For example:

\begin{itemize}
    \item \texttt{Integer a = new Integer(123);}
    \item \texttt{Double b = new Double(3.14);}
    \item \texttt{Character c = new Character('$');}
    \item \texttt{Boolean d = new Boolean(true);}
\end{itemize}

This explicitly creates an object containing the primitive value.

\subsection{Modern Approach: Autoboxing}

The modern and preferred method is autoboxing, where the conversion from primitive to object happens implicitly:

\begin{itemize}
    \item \texttt{Integer a = 123;} // Autoboxing
\end{itemize}

This simplifies code and is the recommended approach.  Wrapper classes provide access to utility methods.  They're essential when using primitive values within collections frameworks like \texttt{ArrayList}.

\subsection{Summary}

This document explains composition in Java using a car and engine example, and details the use of wrapper classes to treat primitive data types as objects.  It emphasizes the modern, autoboxing approach for wrapper classes.


\section{Autoboxing and Unboxing in Java}

This document summarizes autoboxing, unboxing, and utility methods of wrapper classes in Java.

\subsection{Autoboxing}

Autoboxing is the automatic conversion of primitive data types (int, double, char, boolean) into their corresponding wrapper classes (Integer, Double, Character, Boolean).  For example:

\begin{itemize}
    \item \texttt{Integer a = 123;}
    \item \texttt{Double b = 3.14;}
    \item \texttt{Character c = '\$';}
    \item \texttt{Boolean d = true;}
\end{itemize}

This is analogous to creating strings: \texttt{String e = "Pizza";}.  These are all reference data types (objects).

\subsection{Unboxing}

Unboxing is the reverse process: converting a wrapper class object back to its primitive type.  For example:

\begin{itemize}
    \item \texttt{int x = a;}
    \item \texttt{double y = b;}
    \item \texttt{char z = c;}
    \item \texttt{boolean w = d;}
\end{itemize}

\subsection{Wrapper Class Utility Methods}

Wrapper classes provide useful static methods.  Here are some examples:

\subsubsection{\texttt{toString()} Method}

This method converts a primitive data type to its string representation:

\begin{itemize}
    \item \texttt{String strA = Integer.toString(123);}
    \item \texttt{String strB = Double.toString(3.14);}
    \item \texttt{String strC = Character.toString('a');}
    \item \texttt{String strD = Boolean.toString(true);}
\end{itemize}

String concatenation is then possible: \texttt{String x = strA + strB + strC + strD;}

\subsubsection{\texttt{parseXXX()} Methods}

These methods convert strings to primitive types:

\begin{itemize}
    \item \texttt{int a = Integer.parseInt("123");}
    \item \texttt{double b = Double.parseDouble("3.14");}
    \item \texttt{char c = "Pizza".charAt(0);}  (\texttt{charAt()} is a String method, not a wrapper class method)
    \item \texttt{boolean d = Boolean.parseBoolean("true");}
\end{itemize}

Note that \texttt{Character} lacks a \texttt{parse} method;  \texttt{charAt()} is used instead.

\subsubsection{Other Utility Methods}

The \texttt{Character} wrapper class offers additional methods like \texttt{isLetter()} and \texttt{isUpperCase()} for validating user input.  For example:

\begin{itemize}
    \item \texttt{boolean isLetterB = Character.isLetter('b');} // true
    \item \texttt{boolean isUpperCaseB = Character.isUpperCase('b');} // false
\end{itemize}


\section{Wrapper Classes and ArrayLists}

This document summarizes the use of wrapper classes and array lists in Java.

\subsection{Wrapper Classes}

Wrapper classes are used to treat primitive data types as objects.  This is necessary when working with collections like `ArrayLists`, which only accept objects.  Wrapper classes also offer useful static utility methods.

\subsection{ArrayLists}

An `ArrayList` is a dynamic, resizable array that stores objects.  Primitives can be stored using autoboxing (implicitly converting primitives to their wrapper class equivalents).

\begin{itemize}
    \item \textbf{Creating an ArrayList:}  To create an ArrayList, use the following pattern:
    \texttt{ArrayList<Type> list = new ArrayList<>();}  Replace \texttt{Type} with the desired data type (e.g., \texttt{Integer}, \texttt{Double}, \texttt{String}).  The diamond operator (\texttt{<>}) is used for generics.  The import statement is \texttt{import java.util.ArrayList;}.
    \item \textbf{Adding Elements:} Use the \texttt{add()} method.
    \item \textbf{Removing Elements:} Use the \texttt{remove()} method, specifying the index.
    \item \textbf{Setting Elements:} Use the \texttt{set()} method to replace an element at a specific index.
    \item \textbf{Getting Elements:} Use the \texttt{get()} method to retrieve an element at a given index.
    \item \textbf{Getting Size:} Use the \texttt{size()} method to get the number of elements.
    \item \textbf{Sorting:} Use the \texttt{Collections.sort()} method (requiring \texttt{import java.util.Collections;}).
    \item \textbf{Iteration:} Use an enhanced for loop to iterate through the elements.
\end{itemize}

\subsection{Example: User Input and ArrayList}

An example is provided demonstrating how to create an ArrayList, accept user input to populate it with food items, and then process the list.  This example uses a \texttt{Scanner} (requiring \texttt{import java.util.Scanner;}) to obtain user input and remembers to close the \texttt{Scanner} using \texttt{scanner.close()}.  Input buffering is handled by calling \texttt{scanner.nextLine()} after reading an integer.

The example highlights the steps to create an ArrayList of Strings, prompt the user for the number of food items, read the inputs, store them in the ArrayList, and then potentially process the list further.  Error handling (e.g., for non-integer input) is not explicitly shown in this summary.


\section{Using ArrayLists in Java}

This section describes how to use ArrayLists in Java to store and manage a list of food items.  The process involves:

\begin{itemize}
\item Obtaining the number of food items from the user.
\item Using a \texttt{for} loop to iterate and collect the food items.
\item Adding each food item to an ArrayList.
\item Printing the final ArrayList.
\end{itemize}

The code uses a \texttt{Scanner} to get user input and a \texttt{for} loop to iterate a specified number of times. Within the loop, it prompts the user for each food item, adds it to the ArrayList, and finally prints the complete list.  An ArrayList is described as a resizable array capable of storing objects (including primitives via autoboxing).


\section{Exception Handling in Java}

This section covers exception handling in Java, focusing on the \texttt{try-catch} mechanism.  Exceptions, events that disrupt a program's flow, can occur due to various reasons such as division by zero, resource unavailability, or input type mismatches.

\subsection{Handling Arithmetic Exceptions}

The example demonstrates handling an \texttt{ArithmeticException} (e.g., division by zero) using a \texttt{try-catch} block. The "dangerous" code (division by zero) is placed within the \texttt{try} block. If an \texttt{ArithmeticException} occurs, the \texttt{catch} block executes, providing a more graceful error message instead of abruptly terminating the program.

\begin{verbatim}
try {
    int result = 1 / 0; // This will throw an ArithmeticException
} catch (ArithmeticException e) {
    System.out.println("You can't divide by zero, idiot");
}
\end{verbatim}


\subsection{Handling Input Mismatch Exceptions}

Another example showcases handling an \texttt{InputMismatchException}. This exception arises when the program expects a specific data type (e.g., integer) but receives a different one (e.g., a string).  The code prompts the user for a number. If the user enters non-numeric input, the \texttt{catch} block handles the \texttt{InputMismatchException}, allowing the program to continue.

\begin{verbatim}
Scanner scanner = new Scanner(System.in);
try {
    System.out.print("Enter a number: ");
    int number = scanner.nextInt();
    System.out.println("You entered: " + number);
} catch (InputMismatchException e) {
    System.out.println("That wasn't a number. Let's try it again.");
}
\end{verbatim}

The use of a \texttt{Scanner} and the handling of potential exceptions during user input are highlighted as important aspects of robust program design.  Note that the \texttt{import java.util.Scanner;} and \texttt{import java.util.InputMismatchException;} statements are implied but omitted for brevity as per the prompt's instructions.


\section{Exception Handling in Java}

This section summarizes exception handling in Java, including the use of `try`, `catch`, and `finally` blocks.  The text demonstrates how to handle exceptions such as `InputMismatchException` and `IOException`, emphasizing the importance of specific exception handling over a generic `catch-all` approach.  The use of `finally` for resource cleanup (e.g., closing scanners or files) is also highlighted.  The introduction of `try-with-resources` for automatic resource management is mentioned as a modern best practice.

\subsection{Try, Catch, and Finally}

\begin{itemize}
    \item Use a \texttt{try} block to enclose code that might throw an exception.
    \item Use \texttt{catch} blocks to handle specific exception types.  Catching specific exceptions allows for more informative error messages to the user.
    \item A generic \texttt{catch (Exception e)} block can be used as a last resort, but is generally discouraged in favor of specific exception handling.
    \item The \texttt{finally} block is optional but highly recommended for resource cleanup (e.g., closing files, network connections). It always executes, regardless of whether an exception occurred.
    \item Modern Java promotes \texttt{try-with-resources} to automatically manage resources and ensure they are closed even if exceptions occur.
\end{itemize}

\section{File Writing in Java}

This section introduces different methods for writing files in Java, focusing on the \texttt{FileWriter} class for writing small to medium-sized text files.  It also briefly mentions \texttt{BufferedWriter}, \texttt{PrintWriter}, and \texttt{FileOutputStream} for different file types and performance considerations.  The example uses \texttt{FileWriter} within a \texttt{try-catch} block to handle potential \texttt{IOException} exceptions, demonstrating best practices for error handling during file operations.

\subsection{Using FileWriter}

The text provides a basic example of writing to a file using \texttt{FileWriter} in Java.  Key points include:

\begin{itemize}
    \item Creating a \texttt{FileWriter} object with a file path.
    \item Enclosing file writing operations within a \texttt{try} block to handle potential \texttt{IOException} exceptions.
    \item Including a \texttt{catch} block to handle \texttt{IOException} and potentially \texttt{FileNotFoundException}.
    \item Importing necessary classes (\texttt{java.io.FileWriter}, \texttt{java.io.IOException}).
\end{itemize}

The example uses a simple file path relative to the source folder.  The importance of treating file I/O as "dangerous code" and wrapping it in a \texttt{try} block is emphasized.


\section{Writing Files in Java}

This document summarizes how to write and read files in Java.

\subsection{Writing Files}

We use a \texttt{FileWriter} to write to files.  The \texttt{write()} method takes a string as input.  If no exceptions occur, a confirmation message is printed.  The file is created in the source folder by default, or at a specified path (using an absolute path, requiring double forward slashes to escape single forward slashes).

\begin{itemize}
    \item \textbf{File path:} Specify the file location using either a relative or absolute path.  Java interprets a single forward slash as an escape sequence.
    \item \textbf{Exception Handling:} Use a \texttt{try-catch} block to handle exceptions, such as \texttt{IOException} (a general I/O error) and \texttt{FileNotFoundException} (if the file path is invalid). Catch specific exception types first for better error handling.
    \item \textbf{Improved Organization:} Use string variables to store file paths and text content for better code readability.
    \item \textbf{Multi-line Strings:} For multi-line strings, use triple double quotes (\texttt{"""Your text"""}).
\end{itemize}


\subsection{Reading Files}

There are three common methods for reading files in Java:

\begin{itemize}
    \item \textbf{BufferedReader and FileReader:} Best for reading text files line by line. \texttt{BufferedReader} acts as a buffer improving efficiency.
    \item \textbf{FileInputStream:} Best for binary files (images, audio).
    \item \textbf{RandomAccessFile:} Best for reading or writing specific portions of large files.
\end{itemize}

This document primarily demonstrates using \texttt{BufferedReader} and \texttt{FileReader} to read a text file.  The file path is specified using a string variable, including the file name and extension.


The overall text describes how to write and read files in Java, covering exception handling, file path specification, and different approaches for reading various file types.  It emphasizes best practices for code organization and error handling.


\section{Reading Files in Java}

\subsection{Combining \texttt{BufferedReader} and \texttt{FileReader}}

This section explains how to read a file in Java using a \texttt{BufferedReader} and a \texttt{FileReader}.

\begin{itemize}
    \item Create a \texttt{BufferedReader} object: \texttt{BufferedReader reader = new BufferedReader(...)}  This requires importing \texttt{java.io.BufferedReader}.
    \item The \texttt{BufferedReader} constructor requires a \texttt{Reader} object.  We'll use a \texttt{FileReader}.
    \item Create a \texttt{FileReader} object: \texttt{FileReader(filePath)}. This requires importing \texttt{java.io.FileReader}. The \texttt{filePath} is a String representing the file's location.
    \item Use \texttt{try-with-resources} to handle potential exceptions (\texttt{FileNotFoundException}, \texttt{IOException}):
\end{itemize}

\begin{verbatim}
try (BufferedReader reader = new BufferedReader(new FileReader(filePath))) {
    // Code to read the file goes here
} catch (FileNotFoundException e) {
    System.out.println("Could not locate file");
} catch (IOException e) {
    System.out.println("Something went wrong");
}
\end{verbatim}

This requires importing \texttt{java.io.FileNotFoundException} and \texttt{java.io.IOException}.

\begin{itemize}
    \item Read the file line by line using a \texttt{while} loop and the \texttt{readLine()} method:
\end{verbatim}

\begin{verbatim}
String line;
while ((line = reader.readLine()) != null) {
    System.out.println(line);
}
\end{verbatim}

\subsection{Alternative Methods}

Other methods for reading files in Java include \texttt{FileInputStream} and \texttt{RandomAccessFile}.

\section{Creating an Audio Player in Java}

This section discusses creating a simple audio player in Java.

\begin{itemize}
    \item This project is not compatible with MP3 files; conversion to WAV format is necessary.  Online converters are readily available.
    \item For MP3 playback, an external library like JavaFX is required.
    \item You can use YouTube's audio library for free sample music.
    \item After converting an MP3 to a WAV file, place the file in your project's source folder for easy access using a relative file path.
    \item Create a \texttt{File} object using the file path: \texttt{File file = new File(filePath)}.
\end{itemize}


**Summary:** The text describes how to read text files line by line in Java using \texttt{BufferedReader} and \texttt{FileReader}, handling potential exceptions with \texttt{try-with-resources}.  It also briefly touches on creating a simple audio player, noting the need for WAV files or external libraries for MP3 support.


\section{Playing Audio in Java: Handling Exceptions and User Input}

This document summarizes the process of playing audio files in Java, focusing on exception handling and user interaction.  The code utilizes \texttt{AudioInputStream}, \texttt{Clip}, and \texttt{Scanner} objects.

\subsection{Loading and Playing the Audio}

The core functionality involves loading an audio file using \texttt{AudioInputStream} and playing it using a \texttt{Clip} object.  Error handling is crucial, implemented through multiple \texttt{catch} blocks.

\begin{itemize}
    \item \textbf{File Loading:} The program attempts to load an audio file.  A \texttt{FileNotFoundException} is caught if the file is not found.
    \item \textbf{AudioInputStream Creation:} An \texttt{AudioInputStream} object is created.  An \texttt{UnsupportedAudioFileException} is caught if the file format is not supported. An \texttt{IOException} is handled as a general safety net.
    \item \textbf{Clip Creation and Opening:} A \texttt{Clip} object (acting as a player) is created and opened using the \texttt{AudioInputStream}.  A \texttt{LineUnavailableException} is caught if the audio resource is unavailable.
    \item \textbf{Playing Audio:} The \texttt{start()} method plays the audio.  The program originally terminated immediately after starting playback.  To address this, user input is incorporated.
\end{itemize}

\subsection{Exception Handling}

The code employs robust exception handling to manage potential issues:

\begin{itemize}
    \item \texttt{FileNotFoundException}: Handles cases where the specified audio file is not found.
    \item \texttt{UnsupportedAudioFileException}: Handles cases where the audio file format is not supported.
    \item \texttt{LineUnavailableException}: Handles cases where the audio line is unavailable.
    \item \texttt{IOException}: Acts as a general catch-all for other I/O errors.
\end{itemize}

\subsection{User Interaction and Resource Management}

To prevent premature program termination after audio playback starts, a user interface is added.

\begin{itemize}
    \item \textbf{Scanner Input:} A \texttt{Scanner} object is used to accept user input (play, stop, reset, quit).
    \item \textbf{Try-with-resources:} The \texttt{try-with-resources} statement is used to automatically close both the \texttt{AudioInputStream} and the \texttt{Scanner}, ensuring resource cleanup.
\end{itemize}

The final \texttt{finally} block outputs a "bye" message, indicating program completion.  The addition of user input provides control over the audio playback, allowing the user to interact with the program until they choose to quit.


\section{Audio Player in Java}

This section describes the creation of a simple audio player in Java.

\subsection{Program Structure}

\begin{itemize}
\item Uses \texttt{try-with-resources} for automatic resource closure (scanner, audio stream).  \texttt{clip} is handled differently because it doesn't implement \texttt{AutoCloseable}.
\item Main loop continues as long as user input is not 'Q' (quit).
\item User options: P (play), S (stop), R (reset), Q (quit).
\item Uses an enhanced \texttt{switch} statement to handle user input.
\item Error handling for invalid choices.
\end{itemize}

\subsection{Code Snippets}

\begin{itemize}
\item Scanner initialization: \texttt{Scanner scanner = new Scanner(System.in);}
\item Loop condition: \texttt{while (!response.equals(\"Q\"))}
\item User input handling and uppercase conversion: \texttt{response = scanner.next().toUpperCase();}
\item Enhanced switch for handling user actions (P, S, R, Q).  Example: \texttt{case \"P\": clip.start(); break;}
\item Error handling: \texttt{default: System.out.println(\"Invalid Choice\");}
\end{itemize}

\section{Hangman Game in Java}

This section outlines the development of a Hangman game in Java.

\subsection{Game Setup}

\begin{itemize}
\item Uses a \texttt{String} variable (\texttt{word}) to store the secret word (initially "pizza" for testing).
\item Employs a \texttt{Scanner} for user input.  Remember to close the scanner: \texttt{scanner.close()}.
\item Uses an \texttt{ArrayList<Character>} (\texttt{wordState}) to represent the game's state, initially filled with underscores.
\item Tracks incorrect guesses with an integer variable (\texttt{wrongGuesses}).
\item Populates \texttt{wordState} with underscores using a \texttt{for} loop and the \texttt{add} method.
\end{itemize}

\subsection{Further Development (Implied)}

The provided text indicates that later stages of the Hangman game will involve importing a word list and selecting a random word from that list.  This is not detailed in the provided code.



\textbf{Summary:} The text describes the creation of two Java programs: a simple audio player and a Hangman game.  The audio player uses a \texttt{Scanner} for user input, a \texttt{switch} statement for command processing, and handles audio playback using a \texttt{Clip} object. The Hangman game uses a \texttt{Scanner}, an \texttt{ArrayList} to track the guessed letters, and initializes the game board with underscores.  Future development of the Hangman game will involve random word selection from a file.


\section{Java Hangman Game Implementation}

This document summarizes the implementation of a Java Hangman game, focusing on key aspects like word representation, game display, and user interaction.

\subsection{Word State and Display}

\begin{itemize}
    \item The game uses an \texttt{ArrayList} of characters (\texttt{wordState}) to represent the secret word. Initially, it's filled with underscores.
    \item The underscores are displayed to the user, separated by spaces, for a cleaner presentation than the default array output.  This is achieved using an enhanced for loop iterating through \texttt{wordState} and printing each character followed by a space.
\end{itemize}

\subsection{Hangman Art}

\begin{itemize}
    \item A \texttt{getHangmanArt} method generates ASCII art based on the number of incorrect guesses (0-6).
    \item The method uses a \texttt{switch} statement to return different string representations of the hangman image.
    \item Multi-line strings in Java are handled using triple double quotes (\texttt{"\"\"\"}) to define the different stages of the hangman drawing.
    \item  A backslash (\textbackslash{}) is represented using two backslashes (\texttt{\textbackslash{}\textbackslash{}}) in Java strings.
\end{itemize}

\subsection{User Interaction}

\begin{itemize}
    \item The game prompts the user to guess a letter using \texttt{System.in.read()}.
    \item User input is stored in the \texttt{guess} variable.
    \item  Further logic (not detailed in the provided text) would handle letter matching, updating \texttt{wordState}, and managing win/loss conditions.
\end{itemize}

\subsection{Welcome Message}

A welcome message is added to the game, using decorators (asterisks) for visual appeal:  \texttt{Welcome to Java Hangman}.


This implementation uses Java features like \texttt{ArrayLists}, enhanced for loops, \texttt{switch} statements, and multi-line strings to create an interactive Hangman game.  The focus is on clear display of the game state and user-friendly interaction.


\section*{Hangman Game Implementation Summary}

This document summarizes the implementation of a Hangman game.  The game accepts single-character input from the user, converts it to lowercase, and checks if it exists in the target word ("Pizza" in the examples).  The game continues until either the user guesses the word or makes six incorrect guesses.

\subsection*{User Input and Character Processing}

\begin{itemize}
    \item User input is obtained using a scanner.
    \item Input is processed to get the first character and convert it to lowercase.
    \item The first character is obtained using the \texttt{charAt(0)} method.
\end{itemize}

\subsection*{Guess Verification}

\begin{itemize}
    \item The game checks if the guessed character exists in the target word using the \texttt{indexOf()} method.
    \item If \texttt{indexOf()} returns a value greater than or equal to 0, the guess is correct.
    \item A correct guess outputs "Correct guess\textbackslash{}n".
    \item An incorrect guess outputs "Wrong guess\textbackslash{}n" and increments the \texttt{wrongGuesses} counter.
\end{itemize}

\subsection*{Game State Update}

\begin{itemize}
    \item Correct guesses update the \texttt{wordState} to reflect the revealed letters.
    \item A \texttt{for} loop iterates through the word to identify matching characters.
    \item The \texttt{charAt()} method is used to access individual characters in the word.
\end{itemize}

\subsection*{Game Loop and Termination}

\begin{itemize}
    \item A \texttt{while} loop continues the game as long as \texttt{wrongGuesses} is less than 6.
    \item Inside the loop, the game prompts for a guess, verifies it, and updates the game state.
    \item A \texttt{getHangmanArt()} method displays the hangman figure based on the number of incorrect guesses.
\end{itemize}

\subsection*{Game Over Condition}

\begin{itemize}
    \item If \texttt{wrongGuesses} reaches 6 or more, the game ends.
    \item An \texttt{if} statement checks for this condition outside the \texttt{while} loop.
    \item The game over message displays the complete hangman figure and the target word.
\end{itemize}


\section{Hangman Game Improvement: Winning Condition and Word Selection}

\subsection{Winning Condition}

\begin{itemize}
    \item After a correct guess, check if the \texttt{wordState} array list contains any lowercase characters using \texttt{wordState.contains("_")}.
    \item Use the \texttt{!} (not) operator to invert the result. If \texttt{!wordState.contains("_")} is true, the player has guessed all letters.
    \item Display the hangman art one last time using \texttt{system.out.print(getHangmanArt(numberOfWrongGuesses))}.
    \item Display "You win!" and the word.
    \item Use a \texttt{break} statement to exit the \texttt{while} loop.
\end{itemize}


\subsection{Reading Words from a File}

\begin{itemize}
    \item Create a \texttt{words.txt} file containing a list of words, each on a new line (e.g., Apple, orange, banana).
    \item Create a \texttt{String filePath} variable to store the file path (\texttt{"words.txt"}).
    \item Create an \texttt{ArrayList<String> words = new ArrayList<>();} to store the words from the file.
    \item Use a \texttt{try-with-resources} block to handle file reading:
    \begin{itemize}
        \item Create a \texttt{BufferedReader reader = new BufferedReader(new FileReader(filePath));}
        \item Use a \texttt{while} loop to read lines from the file: \texttt{String line = reader.readLine(); while (line != null) \{ ... \}}
        \item Add each line (after trimming whitespace) to the \texttt{words} ArrayList: \texttt{words.add(line.trim());}
        \item Handle \texttt{FileNotFoundException} and \texttt{IOException}.
    \end{itemize}
    \item Output the \texttt{words} ArrayList to verify the file reading.
    \item Use a \texttt{Random} object to select a random word from the \texttt{words} ArrayList: \texttt{words.get(random.nextInt(words.size()))}.
\end{itemize}

\section{Summary}

This text describes improvements to a Hangman game.  First, it details how to determine a win condition by checking if all letters have been guessed. Second, it explains how to load words from a \texttt{words.txt} file into an \texttt{ArrayList} using a \texttt{BufferedReader} and a \texttt{try-with-resources} block for error handling. Finally, it shows how to randomly select a word from this list for the game.


\section{Hangman Game in Java}

The number of words in the hangman game depends on the size of the word list.  The size method returns the number of words (e.g., 8). A random index between 0 and the size-1 is then selected to choose a word. The program outputs the randomly selected word (e.g., "pineapple", "apple", "orange", "coconut").  The user then guesses letters.

\subsection{Key Points}

\begin{itemize}
    \item Word selection is random from a list.
    \item The program handles user input (letter guesses).
    \item The game ends when the word is guessed or all attempts are used.
\end{itemize}


\section{Working with Dates and Times in Java}

This section demonstrates working with dates and times in Java using `LocalDate`, `LocalTime`, `LocalDateTime`, and `Instant`.

\subsection{Retrieving Current Date and Time}

\begin{itemize}
    \item \texttt{LocalDate.now()} returns the current date.
    \item \texttt{LocalTime.now()} returns the current time.
    \item \texttt{LocalDateTime.now()} returns the current date and time.
    \item \texttt{Instant.now()} returns the current UTC timestamp.
\end{itemize}

The output of these methods is often not user-friendly.  Custom formatting is used to improve readability.


\subsection{Custom Date and Time Formatting}

A \texttt{DateTimeFormatter} object is used to format dates and times.  The \texttt{ofPattern()} method takes a string specifying the desired format (e.g., \texttt{MM-dd-yyyy HH:mm:ss}).  Example patterns are shown, allowing flexible formatting for day, month, year, hours, minutes, and seconds.  The \texttt{format()} method applies this format to a \texttt{LocalDateTime} object.

\subsection{Creating Custom Date and Time Objects}

\texttt{LocalDate.of(year, month, day)} creates a \texttt{LocalDate} object.  \texttt{LocalDateTime.of(year, month, day, hour, minute, second)} creates a \texttt{LocalDateTime} object.  Examples show creating objects for specific dates and times, such as Christmas Day at noon.

\subsection{Comparing Dates}

The text mentions that dates can be compared, although the specific code for this is not provided in the excerpt.


\textbf{Summary:} The text provides a basic overview of a hangman game implemented in Java and a detailed explanation of how to work with dates and times in Java using different classes and formatting techniques.  It covers obtaining the current date and time, formatting them for better readability, creating custom date and time objects, and briefly mentions the possibility of date comparison.


\section{Date and Time Comparison in Java}

This section demonstrates date and time comparison using Java.  Two dates, "Christmas 2024" and "New Year's Day 2025", are used as examples.  The code utilizes the `isBefore`, `isAfter`, and `equals` methods to compare dates.

\subsection{Date Comparison Example}

\begin{itemize}
    \item Rename the first date as \texttt{dateOne} and the second as \texttt{dateTwo}.
    \item \texttt{dateTwo} is set to January 1st, 2025, at midnight.
    \item The code compares \texttt{dateOne} and \texttt{dateTwo} using an \texttt{if} statement and the built-in \texttt{isBefore}, \texttt{isAfter}, and \texttt{equals} methods.
    \item Output messages indicate whether \texttt{dateOne} is before, after, or equal to \texttt{dateTwo}.
\end{itemize}

This example showcases basic date comparison functionality, useful for tasks such as scheduling or alarm clock applications.


\section{Anonymous Classes in Java}

This section introduces anonymous classes in Java.  An anonymous class is a class without a name, intended for single-use scenarios where creating a separate class would be cumbersome.

\subsection{Why Use Anonymous Classes?}

\begin{itemize}
    \item Anonymous classes avoid the need to create new classes for objects with only minor differences.
    \item They are frequently used with features like \texttt{TimerTask}, the \texttt{Runnable} interface, and callbacks.
    \item The example uses a \texttt{Dog} class and a \texttt{ScoobyDoo} object which speaks English rather than barking, highlighting the benefits of anonymous classes.
\end{itemize}

\subsection{Creating an Anonymous Class}

\begin{itemize}
    \item To create an anonymous class, add curly braces \{\} after creating an object of a parent class.
    \item Within these braces, override methods or define unique features of this specific object.
    \item The example shows overriding the \texttt{speak} method of a \texttt{Dog} object to make a "Scooby-Doo" object which speaks English.
\end{itemize}

Anonymous classes provide a concise way to add custom behavior without the overhead of creating named classes.  They are particularly useful for one-time uses and situations where creating a new class for each unique behavior would be excessive.

\section{Timers and TimerTasks in Java}

This section briefly introduces the \texttt{Timer} and \texttt{TimerTask} classes in Java, which are used for scheduling tasks at specific times or intervals.  Further details will be covered in a subsequent video.  The \texttt{Timer} class is useful for various applications requiring scheduled updates or repetitive actions like sending notifications.


\section{Using \texttt{Timer} and \texttt{TimerTask} in Java}

\subsection{Scheduling Tasks with \texttt{Timer} and \texttt{TimerTask}}

This section explains how to schedule tasks using Java's \texttt{Timer} and \texttt{TimerTask} classes.  The key points are:

\begin{itemize}
    \item \texttt{Timer} is used to schedule tasks at specific times or periodically.
    \item \texttt{TimerTask} represents the task to be executed.
    \item  \texttt{TimerTask} requires overriding the \texttt{run()} method to define the task's action.  This is often done using an anonymous class for conciseness.
    \item The \texttt{schedule()} method of \texttt{Timer} is used to schedule a \texttt{TimerTask}. It takes the task and a delay (in milliseconds) as arguments.  It can also schedule at a fixed rate.
    \item The \texttt{cancel()} method stops a running \texttt{Timer}.
\end{itemize}

A simple example involves printing "hello" after a 3-second delay:

\texttt{Timer timer = new Timer();} \newline
\texttt{TimerTask task = new TimerTask() \{\newline
    \hspace*{0.5cm} public void run() \{\newline
    \hspace*{1cm} System.out.println("hello");\newline
    \hspace*{0.5cm} \}\newline
\};} \newline
\texttt{timer.schedule(task, 3000);}


For periodic execution,  use \texttt{timer.scheduleAtFixedRate(task, delay, period)}.

To cancel the timer, call \texttt{timer.cancel()}.  This is often done within the \texttt{run()} method itself using a counter to control the number of executions.


\subsection{Countdown Timer Example}

This section outlines creating a countdown timer as a mini-project, illustrating the practical application of \texttt{Timer} and \texttt{TimerTask}. The approach involves:

\begin{itemize}
    \item Accepting user input for the countdown duration (though this is simplified in the initial explanation).
    \item Creating a \texttt{Timer} and \texttt{TimerTask} object.
    \item Overriding the \texttt{run()} method in the \texttt{TimerTask} (again, often using an anonymous class).
    \item Using a loop and counter within the \texttt{run()} method to decrement the countdown and print updates until the counter reaches zero.
\end{itemize}

This demonstrates a more complex use case where the \texttt{TimerTask} handles the countdown logic and potentially user interaction. The example will be refined in a subsequent video to include user input.

\textbf{Summary:} The text describes how to use Java's \texttt{Timer} and \texttt{TimerTask} classes for scheduling tasks, focusing on one-time and periodic execution.  It then outlines a countdown timer example as a more advanced application of these classes, demonstrating how to integrate loops and conditional statements for controlling task execution.  The use of anonymous classes for simplifying \texttt{TimerTask} implementation is highlighted throughout.


\section*{Countdown Timer Program in Java}

This section describes a Java program that creates a countdown timer using \texttt{TimerTask} and \texttt{Timer}.  The program initially uses a hardcoded countdown, then is enhanced to accept user input.

\subsection*{Implementation Details}

\begin{itemize}
    \item An anonymous inner class overrides the \texttt{run()} method of \texttt{TimerTask}.
    \item An integer variable \texttt{count} tracks the countdown.
    \item The \texttt{scheduleAtFixedRate()} method of \texttt{Timer} is used to execute the task repeatedly.
    \item An \texttt{if} statement checks if \texttt{count} is less than zero, displaying "Happy New Year" and canceling the timer.
    \item User input is incorporated using \texttt{Scanner} to specify the countdown duration.
\end{itemize}

The program's core logic involves scheduling a task that decrements a counter and prints the current count. Once the counter reaches zero, a "Happy New Year" message is displayed, and the timer is canceled to prevent infinite execution.  The initial implementation uses a fixed countdown value, which is later replaced with user-provided input.  The use of \texttt{scheduleAtFixedRate} ensures the countdown occurs at a specified interval (1000 milliseconds). The timer is canceled to avoid the program continuing to execute after the countdown is complete.


\section*{Generics in Java}

This section provides a brief overview of generics in Java.

\subsection*{Type Parameters and Type Arguments}

Generics allow writing classes, interfaces, and methods compatible with various data types.  Two key concepts are:

\begin{itemize}
    \item \textbf{Type parameters:}  Represented by angle brackets containing a letter (e.g., \texttt{<T>}), acting as placeholders for actual types.
    \item \textbf{Type arguments:}  Specified within angle brackets when using generic types (e.g., \texttt{<String>}), defining the concrete type used.
\end{itemize}

Type parameters serve as placeholders, while type arguments specify the actual type.  This allows for flexible code that can handle different data types without type casting, improving type safety and code reusability.  An example of generics in action is the \texttt{ArrayList} class, where you specify the type of elements to be stored within the list (e.g., \texttt{ArrayList<String>}). The "diamond operator" allows omitting the type arguments in subsequent uses if the compiler can infer the type.

The text summarizes the creation of a countdown timer in Java and then provides a short explanation of generics in Java, highlighting type parameters and type arguments with the example of `ArrayList`.


\section{Generics in Java: Array Lists and Type Parameters}

This text explains the use of generics in Java, focusing on \texttt{ArrayLists} and custom classes.  It demonstrates how type parameters enhance code reusability and type safety.

\subsection{Array Lists and Type Arguments}

\begin{itemize}
    \item \texttt{ArrayLists} are data structures that can store various data types.
    \item When creating an \texttt{ArrayList}, a type argument must be specified (e.g., \texttt{ArrayList<String>}).  This defines the type of elements the list can hold.
    \item The type parameter (often \texttt{T} or \texttt{E} for element) is declared within angle brackets \texttt{<>} in the class definition.
    \item Changing the type argument changes the type of data the \texttt{ArrayList} can store.
\end{itemize}

\subsection{Creating a Generic Class: The \texttt{Box} Class}

The example introduces a generic class \texttt{Box} to illustrate the power of type parameters.

\begin{itemize}
    \item The \texttt{Box} class uses a type parameter \texttt{T} to represent the type of data it stores.
    \item Methods \texttt{setItem(T item)} and \texttt{getItem()} are defined to add and retrieve items of type \texttt{T}.
    \item When creating a \texttt{Box} object, a type argument must be provided (e.g., \texttt{Box<String>}).
    \item The \texttt{Box} class is reusable; it can store various data types by changing the type argument.
\end{itemize}

\subsection{Using the \texttt{Box} Class}

The example demonstrates using the \texttt{Box} class with different type arguments:

\begin{itemize}
    \item \texttt{Box<String>} is used to store strings.
    \item \texttt{Box<Integer>} is used to store integers.
    \item Attempting to add an incompatible type (e.g., adding a string to \texttt{Box<Integer>}) results in a compilation error.
\end{itemize}

The core idea is that generics allow writing code that works with multiple data types without sacrificing type safety.  The type parameter acts as a placeholder for a specific data type, determined when the generic class or method is used.  This promotes code reusability and reduces the need for multiple, nearly identical versions of the same code for different data types.  The use of type parameters, like `T` or `E`, clarifies the intended type within the generic code.  Raw types are discouraged as they forego the type safety benefits of generics.


\section{Generics in Java}

\subsection{Introduction to Generics}

This section introduces generics in Java, a concept allowing classes, interfaces, and methods to work with various data types.  The logic remains consistent, but the specific type is specified using type parameters and arguments.

\begin{itemize}
    \item Generics enhance code reusability.
    \item Type parameters (e.g., T, U, V) are used to represent unspecified types.
    \item Type arguments are provided during object creation to specify the actual data types.
\end{itemize}

A \texttt{Product} class example is presented, demonstrating the use of two type parameters (\texttt{T} for item, \texttt{U} for price) and a constructor to handle different data types.  Methods \texttt{getItem()} and \texttt{getPrice()} are defined to retrieve the item and price respectively.  Test cases using \texttt{String} and \texttt{double} for one product and \texttt{String} and \texttt{int} for another are shown.


\subsection{Hashmaps in Java}

Hashmaps are key-value pair data structures. Keys must be unique, while values can be duplicated. They are memory-efficient but don't maintain order.  Creating a hashmap requires specifying type parameters for keys (\texttt{K}) and values (\texttt{V}).

\begin{itemize}
    \item Hashmaps store key-value pairs.
    \item Keys must be unique.
    \item Values can be duplicated.
    \item No order is maintained.
    \item Type parameters \texttt{K} and \texttt{V} specify key and value types.
\end{itemize}

An example demonstrates creating a \texttt{HashMap} to store products, using \texttt{String} as the key (item) and \texttt{double} as the value (price). The \texttt{put} method is used to add entries.  The diamond operator (\texttt{<>}) is used to simplify type argument specification.  The example showcases the flexibility of hashmaps in handling different data types while maintaining type safety.


\section{Summary}

This document explains generics and hashmaps in Java. Generics provide type-safe code reusability by allowing classes to work with different data types using type parameters. Hashmaps are efficient key-value stores with unique keys and potentially duplicated values.  Both concepts leverage type parameters to enhance code flexibility and maintain type safety.


\section{Hashmap Operations}

This text describes basic operations on a hashmap (key-value store) using Java-like syntax.  It covers adding, retrieving, updating, removing, and checking the existence of key-value pairs, along with iterating through the map's contents for customized output.

\subsection{Adding and Updating Elements}

\begin{itemize}
    \item Elements are added using the \texttt{put} method, taking a key and a value as arguments.
    \item Keys must be unique; attempting to add a duplicate key overwrites the existing key-value pair.
    \item Example:  Adding an apple (\$0.50), orange (\$0.75), banana (\$0.25), and coconut (\$1.00).
\end{itemize}

\subsection{Retrieving and Removing Elements}

\begin{itemize}
    \item The \texttt{get} method retrieves the value associated with a specific key.
    \item The \texttt{remove} method removes a key-value pair given the key.
\end{itemize}

\subsection{Checking for Key and Value Existence}

\begin{itemize}
    \item The \texttt{containsKey} method checks if a key exists (returns a Boolean).
    \item The \texttt{containsValue} method checks if a value exists (returns a Boolean;  data type matters).
\end{itemize}

\subsection{Getting Map Size and Iterating Through Key-Value Pairs}

\begin{itemize}
    \item The \texttt{size} method returns the number of key-value pairs in the map.
    \item An enhanced \texttt{for} loop, combined with the \texttt{keySet} method, allows iteration through keys and retrieval of values using the \texttt{get} method for customized output.
\end{itemize}

\subsection{Example Code Snippet (Conceptual): }

The following demonstrates a conceptual example of iterating and displaying the hashmap contents using an enhanced for loop and the \texttt{keySet} method.  Note this is not valid LaTeX code, but represents the Java-like code discussed.

```java
for (String key : map.keySet()) {
    System.out.println(key + ": $" + map.get(key));
}
```

This code iterates through each key in the map's key set, and for each key, it prints the key and its corresponding value (obtained using the get method), resulting in a cleaner output than directly printing the map.



\section{Hashmaps and Enums in Java}

\subsection{Hashmaps}

A hashmap is a data structure that stores key-value pairs.  Keys must be unique, but values can be duplicated.  The type of both key and value must be specified.

\subsection{Enums (Enumerations)}

Enums are a special kind of class representing a fixed set of constants.  They improve code readability and maintainability, offering efficiency with switches compared to string comparisons.

\begin{itemize}
    \item Enums declare constants.
    \item Convention:  Constants are uppercase.
    \item More efficient in switch statements than strings.
\end{itemize}

\subsection{Enum Example: Days of the Week}

A demonstration creates an enum called \texttt{Day}:

\begin{verbatim}
public enum Day {
    SUNDAY(1), MONDAY(2), TUESDAY(3), WEDNESDAY(4), THURSDAY(5), FRIDAY(6), SATURDAY(7);

    private final int dayNumber;

    Day(int dayNumber) {
        this.dayNumber = dayNumber;
    }

    public int getDayNumber() {
        return this.dayNumber;
    }
}
\end{verbatim}

This enum defines constants for each day of the week, each associated with an integer value.  A constructor initializes the \texttt{dayNumber} field. A getter method, \texttt{getDayNumber()}, is provided for accessing this value.

\subsection{Working with Enums}

Accessing enum constants:

\begin{verbatim}
Day day = Day.SUNDAY;
System.out.println(day); // Output: SUNDAY
System.out.println(day.getDayNumber()); // Output: 1
\end{verbatim}

Using enums in a switch statement (enhanced switch):

\begin{verbatim}
switch (day) {
    case MONDAY, TUESDAY, WEDNESDAY, THURSDAY, FRIDAY -> System.out.println("It is a weekday");
    case SATURDAY, SUNDAY -> System.out.println("It is the weekend");
}
\end{verbatim}

This is more efficient than using strings in a switch.


\subsection{User Input and Enum}

An example using user input to select a day:

\begin{verbatim}
Scanner scanner = new Scanner(System.in);
System.out.print("Enter a day of the week: ");
String response = scanner.nextLine().toUpperCase();
Day day = Day.valueOf(response);
scanner.close();
\end{verbatim}

This code takes user input, converts it to uppercase, and then uses \texttt{Day.valueOf()} to obtain the corresponding enum constant.  Remember to close the \texttt{Scanner} to prevent resource leaks.


\section{Summary}

This document explains hashmaps and enums in Java.  Hashmaps store key-value pairs, while enums represent a fixed set of named constants, improving code readability and efficiency, especially within switch statements. The example demonstrates creating and using an enum for days of the week, highlighting its advantages over using strings for similar purposes.  The code examples show how to create enums, access their values, and use them effectively in switch statements and with user input.


\section{Enums in Java}

This section summarizes the use of enums (enumerations) in Java.  Enums are a special type of class representing a fixed set of constants. They enhance code readability and reliability, particularly beneficial when used with switch statements.  Using enums with switches is faster than using strings.  While integers could be used as an alternative, enums provide significantly better readability.

\subsection{Example: Days of the Week}

The example demonstrates a scenario where enums are used to represent days of the week.  The code handles user input, checking for valid days and providing appropriate feedback.  The example also shows how to add new enum values (e.g., "Pizza Day") and handle exceptions gracefully using try-catch blocks.

\begin{itemize}
    \item Enums improve code readability.
    \item Using enums with switches is faster than using strings.
    \item Exception handling (e.g., \textbackslash{}IllegalArgumentException) is crucial for robust code.
\end{itemize}


\section{Threading in Java}

This section explains the concept of threading in Java, a mechanism for running multiple tasks concurrently to improve performance.  It focuses on the advantages of using threads for time-consuming operations, such as file I/O and network communications.  Two primary methods for creating threads are discussed: extending the \texttt{Thread} class and implementing the \texttt{Runnable} interface.  The latter is preferred for its flexibility, allowing multiple inheritance.


\subsection{Demonstration:  Timer Example}

The example illustrates the benefits of threading using a simple timer scenario.  A user is given 5 seconds to enter their name.  Without threading (using only the main thread), the timer runs to completion *before* the user is prompted for input.  This highlights the issue of blocking operations on the main thread.

\begin{itemize}
    \item Threading allows parallel execution of tasks.
    \item Time-consuming operations (e.g., timers) should run on separate threads to avoid blocking the main thread.
    \item  The \texttt{Thread.sleep()} method is used to simulate delays, but requires a \texttt{try-catch} block to handle \texttt{InterruptedException}.
\end{itemize}

The example shows that using a single thread (the main thread) for both the timer and user input results in a poor user experience.  Threading would be necessary to handle these tasks concurrently and provide a more responsive application.


\section{Introduction to Multithreading in Java}

This document summarizes the creation and use of threads in Java to perform concurrent tasks.  The core concept revolves around creating a separate thread to handle time-consuming operations without blocking the main thread.

\subsection{Creating and Using Threads}

The primary method involves implementing the `Runnable` interface.  This avoids the limitations of single inheritance (inheriting from `Thread`).

\begin{itemize}
    \item Create a new Java class implementing the `Runnable` interface.
    \item Override the `run()` method. This method contains the code to execute in the background thread.
    \item Create an object of this `Runnable` class.
    \item Pass this object to the `Thread` class constructor to create a new `Thread` object.
    \item Call the `start()` method on the `Thread` object to begin execution.
\end{itemize}

The example demonstrates a countdown timer running in a separate thread while the main thread accepts user input.  The main thread waits for either user input or the timer to expire.

\subsection{Daemon Threads}

To ensure the program exits cleanly when the main thread completes, even if background threads are still running, use daemon threads:

\begin{itemize}
    \item Call the `setDaemon(true)` method on the `Thread` object.  This designates it as a daemon thread.
    \item Daemon threads terminate automatically when the main thread finishes.
\end{itemize}

Alternatively, the program can be explicitly terminated within the `run()` method using `System.exit(0)`.

\subsection{Benefits of Multithreading}

Multithreading offers significant advantages:

\begin{itemize}
    \item \textbf{Improved Performance:} Handles time-consuming operations concurrently.
    \item \textbf{Responsiveness:} Maintains user interface responsiveness even during lengthy tasks.
    \item \textbf{Parallelism:} Enables simultaneous execution of multiple tasks.
\end{itemize}

Useful applications include file I/O, network communication, and other background processes.  The example highlights how multithreading can be used to manage user input while simultaneously running a countdown timer, preventing the program from freezing while waiting.


\section{Multi-threading in Java}

\subsection{A Simple Example: Counting to Five}

The core concept of overriding the `run()` method is explained.  A basic example of counting to five with a `for` loop, incorporating a 1-second delay using `Thread.sleep()`, and handling potential `InterruptedException`s is detailed. The code snippet demonstrates creating a `Runnable` object, passing it to a `Thread` object, and starting the thread using `thread.start()`.  An alternative approach using an anonymous `Runnable` object is also shown.

\begin{itemize}
    \item Overriding the `public void run()` method.
    \item Using a `for` loop to count to five.
    \item Introducing a 1-second delay using `Thread.sleep(1000)`.
    \item Handling `InterruptedException` with `try-catch` blocks.
    \item Creating and starting a thread using `Runnable` and `Thread` objects.
    \item Using anonymous `Runnable` objects for conciseness.
\end{itemize}


\subsection{Multi-threading with Two Threads}

The example is extended to demonstrate concurrent execution of two threads, both counting to five.  Getting the name of each thread using `Thread.currentThread().getName()` is also covered.

\begin{itemize}
    \item Creating and starting two threads.
    \item Running threads concurrently.
    \item Accessing thread names using `Thread.currentThread().getName()`.
\end{itemize}


\subsection{Ping-Pong Example}

This section demonstrates a more sophisticated example where two threads exchange "ping" and "pong" messages.  This involves modifying the `Runnable` class to accept and print custom text.

\begin{itemize}
    \item Modifying the `Runnable` class to accept custom text.
    \item Passing text ("ping" and "pong") to the threads.
    \item Coordinating the output between two threads.
\end{itemize}


\subsection{Waiting for Threads to Finish}

The importance of waiting for child threads to complete before the main thread exits is highlighted.  This is achieved by using the `join()` method on each thread, with appropriate `try-catch` blocks for handling `InterruptedException`.

\begin{itemize}
    \item Using `thread.join()` to wait for threads to complete.
    \item Handling `InterruptedException` in the main thread.
\end{itemize}


\section{Summary}

This document provides a basic introduction to multi-threading in Java, demonstrating how to create and manage multiple threads concurrently.  Key concepts like overriding the `run()` method, using `Runnable` and `Thread` objects, handling exceptions, and coordinating thread execution are explained with practical examples.  The examples progress from simple counting to more complex scenarios involving communication between threads and ensuring proper termination.  A brief mention of an alarm clock project as a future application is included.


\section{Setting an Alarm Time}

\subsection{Parsing User Input}

To set the alarm based on user input, we need to parse a string representation of the time into a \texttt{LocalTime} object.  This requires a \texttt{DateTimeFormatter} object.

\begin{itemize}
    \item Import the \texttt{DateTimeFormatter} class.
    \item Create a \texttt{DateTimeFormatter} object: \texttt{formatter = DateTimeFormatter.ofPattern("HH:mm:ss")}.
    \item Prompt the user to enter the alarm time in HH:mm:ss format.
    \item Read user input using a scanner (\texttt{inputTime = scanner.nextLine()}).
    \item Parse the input string into a \texttt{LocalTime} object: \texttt{alarmTime = LocalTime.parse(inputTime, formatter)}.
    \item Output a confirmation message: \texttt{System.out.println("Alarm set for " + alarmTime)}.
\end{itemize}

\subsection{Exception Handling}

Invalid user input (e.g., "Pizza") will cause a \texttt{DateTimeParseException}. We handle this using a \texttt{try-catch} block.

\begin{itemize}
    \item Enclose the parsing code within a \texttt{try} block.
    \item Catch \texttt{DateTimeParseException} and output an error message.
    \item Re-prompt the user for input if an exception occurs.  A \texttt{while} loop ensures valid input before proceeding. The loop continues while \texttt{alarmTime} is \texttt{null}.
\end{itemize}


\section{Creating the Alarm Clock Class}

The alarm clock functionality is implemented in a separate class, \texttt{AlarmClock}, which implements the \texttt{Runnable} interface for threading.

\subsection{AlarmClock Class}

\begin{itemize}
    \item The \texttt{AlarmClock} class has a private, final \texttt{LocalTime} field, \texttt{alarmTime}.
    \item A constructor takes a \texttt{LocalTime} object as input and assigns it to the \texttt{alarmTime} field.
    \item The \texttt{run()} method (required by \texttt{Runnable}) will contain the code to execute when the alarm goes off (not implemented in this summary).
\end{itemize}

\subsection{Integrating AlarmClock}

In the main Java file, after obtaining a valid \texttt{alarmTime}, an \texttt{AlarmClock} object is created and a new thread is started using this object.

\begin{itemize}
    \item Create an \texttt{AlarmClock} object: \texttt{alarmClock = new AlarmClock(alarmTime)}.
    \item Create a new thread: \texttt{alarmThread = new Thread(alarmClock)}.
    \item Start the thread: \texttt{alarmThread.start()}.
\end{itemize}

This completes the basic structure of setting and managing an alarm using Java.  The \texttt{run()} method of the \texttt{AlarmClock} class needs to be implemented to handle the alarm trigger mechanism.  Note that error handling and input validation are crucial for robustness.


\section{Alarm Clock Thread Implementation}

This document summarizes the implementation of an alarm clock using threads in Java. The core functionality involves creating a thread that continuously checks the current time against a set alarm time.

\subsection{Thread Initialization and Time Acquisition}

The main thread starts an alarm thread by calling its \texttt{start()} method. The alarm thread executes its \texttt{run()} method, which begins by obtaining the current time using \texttt{LocalTime.now()}.  This time is stored in a \texttt{LocalTime} object called \texttt{now}.

\subsection{Time Comparison and Loop}

The core logic involves comparing the current time to the alarm time.  A \texttt{while} loop continuously checks if the current time (\texttt{now}) is before the alarm time using the \texttt{isBefore()} method.

\begin{itemize}
    \item If the current time is before the alarm time, the thread sleeps for 1 second using \texttt{Thread.sleep(1000)}.
    \item This sleep operation is wrapped in a \texttt{try-catch} block to handle \texttt{InterruptedException}.
\end{itemize}

\subsection{Time Formatting and Display}

Initially, the current time is displayed with milliseconds.  To improve readability, the code is modified to display the time in \texttt{HH:MM:SS} format using \texttt{printf} with format specifiers (\texttt{\%02d}) to ensure leading zeros for single-digit hours, minutes, or seconds.

\begin{itemize}
    \item The \texttt{\textbackslash r} (carriage return) escape sequence is used to update the time in place on the console, rather than printing a new line for each update.
\end{itemize}

\subsection{Code Optimization}

The code is further optimized by reducing redundant calls to \texttt{LocalTime.now()}. Instead of multiple calls, the current time is obtained once and stored in a \texttt{LocalTime} object which is then used to extract hours, minutes, and seconds.


\subsection{Alarm Trigger}

Once the \texttt{while} loop condition (\texttt{now.isBefore(alarmTime)}) becomes false (meaning the current time is at or past the alarm time), the program outputs "alarm noises" (although audio implementation is not included in this example).


The implementation uses basic time comparison and thread management in Java to create a simple alarm clock functionality.  Error handling for \texttt{InterruptedException} is included, and the output is formatted for better readability.


\section{Alarm Clock Implementation}

\subsection{Basic Beep Alarm}

The initial alarm implementation displays alarm noises and exits.  A single beep is used for testing.  To generate the beep sound in Java:

\begin{itemize}
    \item Access the \texttt{Toolkit} class.
    \item Call the \texttt{getDefaultToolkit()} method.
    \item Call the \texttt{beep()} method.
    \item Import the necessary class.
\end{itemize}

After the alarm time is reached, alarm noises are displayed and a beep sound is produced.


\subsection{Advanced Audio File Playback}

For a more sophisticated alarm, an audio file (WAV) is played.  MP3 files require conversion to WAV beforehand.  The WAV file is placed in the project folder.

\begin{itemize}
    \item A \texttt{String} variable \texttt{filePath} stores the file path (relative or absolute).
    \item The \texttt{AlarmClock} class is modified to accept a \texttt{filePath} as a constructor parameter.
    \item A private method \texttt{playSound()} handles audio playback.
\end{itemize}

The \texttt{playSound()} method utilizes:

\begin{itemize}
    \item \texttt{File} object to represent the audio file.
    \item \texttt{AudioInputStream} to read the audio data.
    \item \texttt{Clip} object to play the audio stream.
    \item \texttt{try-with-resources} to ensure resource closure.
    \item Exception handling for unsupported audio formats (\texttt{UnsupportedAudioFileException}), unavailable audio (\texttt{LineUnavailableException}), and general I/O errors (\texttt{IOException}).
\end{itemize}

The \texttt{Clip.start()} method initiates playback.  A short sleep is initially used to prevent immediate program termination, later replaced with user input.


\subsection{User Input for Alarm Termination}

The initial implementation's 5-second sleep is replaced with a mechanism to stop the alarm when the user presses Enter.

\begin{itemize}
    \item The sleep method is removed.
    \item A \texttt{Scanner} is used to wait for user input.  The Scanner is closed within the \texttt{AlarmClock} class to prevent resource conflicts with other parts of the program using \texttt{System.in}.
\end{itemize}

This allows for a more interactive alarm experience.  The user can stop the alarm manually instead of the alarm automatically stopping after a fixed duration.


\section{Summary}

This document details the implementation of a Java alarm clock, progressing from a simple beep-based alarm to a more advanced version that plays audio files.  The implementation incorporates robust exception handling and user input to improve functionality and user experience.  The use of \texttt{try-with-resources} ensures proper resource management, and careful handling of \texttt{Scanner} prevents potential conflicts.


\section{Java Alarm Clock Implementation}

\subsection{Passing the Scanner Object}

\begin{itemize}
    \item The scanner object will be passed as an argument to the `AlarmClock` class.
    \item This avoids creating multiple scanner instances.
    \item The scanner will be reused within the `AlarmClock` class.
\end{itemize}

\subsection{AlarmClock Class Setup}

\begin{itemize}
    \item A private final `Scanner` object (named `scanner`) will be created within the `AlarmClock` class.
    \item The constructor will receive a `Scanner` object as a parameter.
    \item  `this.scanner = scanner;` assigns the received `Scanner` object to the class's private `scanner` object.
    \item This ensures consistent access to the same `Scanner` instance throughout the `AlarmClock` class's lifecycle.  The `Scanner` is not closed in `main` anymore.
\end{itemize}


\subsection{User Input and Alarm Termination}

\begin{itemize}
    \item  A prompt (\texttt{print("Press enter to stop the alarm")}) will be displayed to the user.
    \item The program waits for user input using \texttt{scanner.nextLine()}.
    \item  The alarm music continues playing until the user presses Enter.
    \item Once Enter is pressed, the music clip's \texttt{stop()} method is called.
    \item Finally, the \texttt{scanner.close()} method is called to release resources.
\end{itemize}


\subsection{Program Execution and Summary}

\begin{itemize}
    \item The alarm clock functionality is demonstrated by setting an alarm time (e.g., 8:07).
    \item Pressing Enter successfully stops the alarm and exits the program.
    \item The example demonstrates a functional alarm clock application written in Java.
\end{itemize}


\section{Detailed Notes Summary (For Expanded A4 Notes)}

To expand this into 10-15 pages of A4 notes (double-sided), consider these points for elaboration:

\textbf{I.  Detailed Code Explanation (Multiple Pages): }

* **Full Java Code:** Provide the complete, well-commented Java code for the `AlarmClock` class and the main application.  Include error handling (e.g., try-catch blocks for file I/O).
* **Class Design:**  Discuss the design choices, including the reasons for using a private final Scanner and passing the Scanner as an argument. Explore alternative design patterns.
* **Constructor Details:** Explain the constructor's role in initializing the `AlarmClock` object, emphasizing the importance of receiving the `Scanner` object correctly.
* **Method Breakdown:**  Provide a detailed explanation of each method in the `AlarmClock` class, focusing on functionality and logic. Include flowcharts or pseudocode for complex methods.
* **Variable Scope and Lifetime:**  Carefully explain the scope and lifetime of each variable within the `AlarmClock` class and the main method, including the scanner object.
* **Resource Management:** Explain the importance of closing the scanner using \texttt{scanner.close()} to prevent resource leaks. Detail the consequences of not closing the scanner.
* **Exception Handling:**  Include detailed examples of how to handle potential exceptions, such as \texttt{FileNotFoundException} if the sound file isn't found.

\textbf{II.  Java Concepts (Multiple Pages):}

* **Object-Oriented Programming (OOP) Principles:** Relate the code to core OOP principles: encapsulation, inheritance (if applicable), polymorphism.
* **Input/Output (I/O):** Explain the use of the `Scanner` class for user input and other potential I/O methods.
* **File Handling:** Explain how to handle audio files (loading, playing, stopping) using appropriate Java libraries. Include details on file paths, handling different audio formats, etc.
* **Threading (Optional):**  If the alarm plays in a separate thread, describe the concepts of threads and concurrency in Java. Discuss thread safety if needed.
* **Error Handling and Debugging:** Discuss debugging techniques and common errors in Java, especially those relevant to this project. Include examples of debugging strategies.
* **Testing and Validation:**  Detail how the code was tested, including unit tests or other verification methods.  Describe how the application was validated to ensure it meets requirements.


\textbf{III.  Project Management Aspects (Few Pages):}


* **Project Requirements:**  Clearly define the initial requirements for the alarm clock program.
* **Design Process:**  Detail the design process followed, including diagrams (class diagrams, sequence diagrams) illustrating the system's architecture and interactions.
* **Testing Strategy:** Document the testing strategy, including types of tests (unit, integration, system), and test results.
* **Future Enhancements:**  Discuss potential future enhancements (e.g., multiple alarms, snooze function, GUI).
* **Code Style and Readability:** Discuss the importance of using consistent code style and best practices for readability and maintainability.

\textbf{IV. Appendix (Few Pages): }

* **Java API References:** Include relevant API documentation for used classes (Scanner, File, etc.).
* **Error Logs (if applicable):** Include error logs from testing or debugging.
* **Glossary of Terms:** Define key programming terms used throughout the notes.



By expanding on these points, you can create comprehensive notes suitable for future reference. Remember to use diagrams, code snippets, and clear explanations to make the notes easy to understand and use.


\end{document}